\documentclass[11pt]{article}

    \usepackage[breakable]{tcolorbox}
    \usepackage{parskip} % Stop auto-indenting (to mimic markdown behaviour)
    
    \usepackage{iftex}
    \ifPDFTeX
    	\usepackage[T1]{fontenc}
    	\usepackage{mathpazo}
    \else
    	\usepackage{fontspec}
    \fi

    % Basic figure setup, for now with no caption control since it's done
    % automatically by Pandoc (which extracts ![](path) syntax from Markdown).
    \usepackage{graphicx}
    % Maintain compatibility with old templates. Remove in nbconvert 6.0
    \let\Oldincludegraphics\includegraphics
    % Ensure that by default, figures have no caption (until we provide a
    % proper Figure object with a Caption API and a way to capture that
    % in the conversion process - todo).
    \usepackage{caption}
    \DeclareCaptionFormat{nocaption}{}
    \captionsetup{format=nocaption,aboveskip=0pt,belowskip=0pt}

    \usepackage[Export]{adjustbox} % Used to constrain images to a maximum size
    \adjustboxset{max size={0.9\linewidth}{0.9\paperheight}}
    \usepackage{float}
    \floatplacement{figure}{H} % forces figures to be placed at the correct location
    \usepackage{xcolor} % Allow colors to be defined
    \usepackage{enumerate} % Needed for markdown enumerations to work
    \usepackage{geometry} % Used to adjust the document margins
    \usepackage{amsmath} % Equations
    \usepackage{amssymb} % Equations
    \usepackage{textcomp} % defines textquotesingle
    % Hack from http://tex.stackexchange.com/a/47451/13684:
    \AtBeginDocument{%
        \def\PYZsq{\textquotesingle}% Upright quotes in Pygmentized code
    }
    \usepackage{upquote} % Upright quotes for verbatim code
    \usepackage{eurosym} % defines \euro
    \usepackage[mathletters]{ucs} % Extended unicode (utf-8) support
    \usepackage{fancyvrb} % verbatim replacement that allows latex
    \usepackage{grffile} % extends the file name processing of package graphics 
                         % to support a larger range
    \makeatletter % fix for grffile with XeLaTeX
    \def\Gread@@xetex#1{%
      \IfFileExists{"\Gin@base".bb}%
      {\Gread@eps{\Gin@base.bb}}%
      {\Gread@@xetex@aux#1}%
    }
    \makeatother

    % The hyperref package gives us a pdf with properly built
    % internal navigation ('pdf bookmarks' for the table of contents,
    % internal cross-reference links, web links for URLs, etc.)
    \usepackage{hyperref}
    % The default LaTeX title has an obnoxious amount of whitespace. By default,
    % titling removes some of it. It also provides customization options.
    \usepackage{titling}
    \usepackage{longtable} % longtable support required by pandoc >1.10
    \usepackage{booktabs}  % table support for pandoc > 1.12.2
    \usepackage[inline]{enumitem} % IRkernel/repr support (it uses the enumerate* environment)
    \usepackage[normalem]{ulem} % ulem is needed to support strikethroughs (\sout)
                                % normalem makes italics be italics, not underlines
    \usepackage{mathrsfs}
    

    
    % Colors for the hyperref package
    \definecolor{urlcolor}{rgb}{0,.145,.698}
    \definecolor{linkcolor}{rgb}{.71,0.21,0.01}
    \definecolor{citecolor}{rgb}{.12,.54,.11}

    % ANSI colors
    \definecolor{ansi-black}{HTML}{3E424D}
    \definecolor{ansi-black-intense}{HTML}{282C36}
    \definecolor{ansi-red}{HTML}{E75C58}
    \definecolor{ansi-red-intense}{HTML}{B22B31}
    \definecolor{ansi-green}{HTML}{00A250}
    \definecolor{ansi-green-intense}{HTML}{007427}
    \definecolor{ansi-yellow}{HTML}{DDB62B}
    \definecolor{ansi-yellow-intense}{HTML}{B27D12}
    \definecolor{ansi-blue}{HTML}{208FFB}
    \definecolor{ansi-blue-intense}{HTML}{0065CA}
    \definecolor{ansi-magenta}{HTML}{D160C4}
    \definecolor{ansi-magenta-intense}{HTML}{A03196}
    \definecolor{ansi-cyan}{HTML}{60C6C8}
    \definecolor{ansi-cyan-intense}{HTML}{258F8F}
    \definecolor{ansi-white}{HTML}{C5C1B4}
    \definecolor{ansi-white-intense}{HTML}{A1A6B2}
    \definecolor{ansi-default-inverse-fg}{HTML}{FFFFFF}
    \definecolor{ansi-default-inverse-bg}{HTML}{000000}

    % commands and environments needed by pandoc snippets
    % extracted from the output of `pandoc -s`
    \providecommand{\tightlist}{%
      \setlength{\itemsep}{0pt}\setlength{\parskip}{0pt}}
    \DefineVerbatimEnvironment{Highlighting}{Verbatim}{commandchars=\\\{\}}
    % Add ',fontsize=\small' for more characters per line
    \newenvironment{Shaded}{}{}
    \newcommand{\KeywordTok}[1]{\textcolor[rgb]{0.00,0.44,0.13}{\textbf{{#1}}}}
    \newcommand{\DataTypeTok}[1]{\textcolor[rgb]{0.56,0.13,0.00}{{#1}}}
    \newcommand{\DecValTok}[1]{\textcolor[rgb]{0.25,0.63,0.44}{{#1}}}
    \newcommand{\BaseNTok}[1]{\textcolor[rgb]{0.25,0.63,0.44}{{#1}}}
    \newcommand{\FloatTok}[1]{\textcolor[rgb]{0.25,0.63,0.44}{{#1}}}
    \newcommand{\CharTok}[1]{\textcolor[rgb]{0.25,0.44,0.63}{{#1}}}
    \newcommand{\StringTok}[1]{\textcolor[rgb]{0.25,0.44,0.63}{{#1}}}
    \newcommand{\CommentTok}[1]{\textcolor[rgb]{0.38,0.63,0.69}{\textit{{#1}}}}
    \newcommand{\OtherTok}[1]{\textcolor[rgb]{0.00,0.44,0.13}{{#1}}}
    \newcommand{\AlertTok}[1]{\textcolor[rgb]{1.00,0.00,0.00}{\textbf{{#1}}}}
    \newcommand{\FunctionTok}[1]{\textcolor[rgb]{0.02,0.16,0.49}{{#1}}}
    \newcommand{\RegionMarkerTok}[1]{{#1}}
    \newcommand{\ErrorTok}[1]{\textcolor[rgb]{1.00,0.00,0.00}{\textbf{{#1}}}}
    \newcommand{\NormalTok}[1]{{#1}}
    
    % Additional commands for more recent versions of Pandoc
    \newcommand{\ConstantTok}[1]{\textcolor[rgb]{0.53,0.00,0.00}{{#1}}}
    \newcommand{\SpecialCharTok}[1]{\textcolor[rgb]{0.25,0.44,0.63}{{#1}}}
    \newcommand{\VerbatimStringTok}[1]{\textcolor[rgb]{0.25,0.44,0.63}{{#1}}}
    \newcommand{\SpecialStringTok}[1]{\textcolor[rgb]{0.73,0.40,0.53}{{#1}}}
    \newcommand{\ImportTok}[1]{{#1}}
    \newcommand{\DocumentationTok}[1]{\textcolor[rgb]{0.73,0.13,0.13}{\textit{{#1}}}}
    \newcommand{\AnnotationTok}[1]{\textcolor[rgb]{0.38,0.63,0.69}{\textbf{\textit{{#1}}}}}
    \newcommand{\CommentVarTok}[1]{\textcolor[rgb]{0.38,0.63,0.69}{\textbf{\textit{{#1}}}}}
    \newcommand{\VariableTok}[1]{\textcolor[rgb]{0.10,0.09,0.49}{{#1}}}
    \newcommand{\ControlFlowTok}[1]{\textcolor[rgb]{0.00,0.44,0.13}{\textbf{{#1}}}}
    \newcommand{\OperatorTok}[1]{\textcolor[rgb]{0.40,0.40,0.40}{{#1}}}
    \newcommand{\BuiltInTok}[1]{{#1}}
    \newcommand{\ExtensionTok}[1]{{#1}}
    \newcommand{\PreprocessorTok}[1]{\textcolor[rgb]{0.74,0.48,0.00}{{#1}}}
    \newcommand{\AttributeTok}[1]{\textcolor[rgb]{0.49,0.56,0.16}{{#1}}}
    \newcommand{\InformationTok}[1]{\textcolor[rgb]{0.38,0.63,0.69}{\textbf{\textit{{#1}}}}}
    \newcommand{\WarningTok}[1]{\textcolor[rgb]{0.38,0.63,0.69}{\textbf{\textit{{#1}}}}}
    
    
    % Define a nice break command that doesn't care if a line doesn't already
    % exist.
    \def\br{\hspace*{\fill} \\* }
    % Math Jax compatibility definitions
    \def\gt{>}
    \def\lt{<}
    \let\Oldtex\TeX
    \let\Oldlatex\LaTeX
    \renewcommand{\TeX}{\textrm{\Oldtex}}
    \renewcommand{\LaTeX}{\textrm{\Oldlatex}}
    % Document parameters
    % Document title
    \title{github\_taruma\_demo\_lstm\_rr}
    
    
    
    
    
% Pygments definitions
\makeatletter
\def\PY@reset{\let\PY@it=\relax \let\PY@bf=\relax%
    \let\PY@ul=\relax \let\PY@tc=\relax%
    \let\PY@bc=\relax \let\PY@ff=\relax}
\def\PY@tok#1{\csname PY@tok@#1\endcsname}
\def\PY@toks#1+{\ifx\relax#1\empty\else%
    \PY@tok{#1}\expandafter\PY@toks\fi}
\def\PY@do#1{\PY@bc{\PY@tc{\PY@ul{%
    \PY@it{\PY@bf{\PY@ff{#1}}}}}}}
\def\PY#1#2{\PY@reset\PY@toks#1+\relax+\PY@do{#2}}

\expandafter\def\csname PY@tok@w\endcsname{\def\PY@tc##1{\textcolor[rgb]{0.73,0.73,0.73}{##1}}}
\expandafter\def\csname PY@tok@c\endcsname{\let\PY@it=\textit\def\PY@tc##1{\textcolor[rgb]{0.25,0.50,0.50}{##1}}}
\expandafter\def\csname PY@tok@cp\endcsname{\def\PY@tc##1{\textcolor[rgb]{0.74,0.48,0.00}{##1}}}
\expandafter\def\csname PY@tok@k\endcsname{\let\PY@bf=\textbf\def\PY@tc##1{\textcolor[rgb]{0.00,0.50,0.00}{##1}}}
\expandafter\def\csname PY@tok@kp\endcsname{\def\PY@tc##1{\textcolor[rgb]{0.00,0.50,0.00}{##1}}}
\expandafter\def\csname PY@tok@kt\endcsname{\def\PY@tc##1{\textcolor[rgb]{0.69,0.00,0.25}{##1}}}
\expandafter\def\csname PY@tok@o\endcsname{\def\PY@tc##1{\textcolor[rgb]{0.40,0.40,0.40}{##1}}}
\expandafter\def\csname PY@tok@ow\endcsname{\let\PY@bf=\textbf\def\PY@tc##1{\textcolor[rgb]{0.67,0.13,1.00}{##1}}}
\expandafter\def\csname PY@tok@nb\endcsname{\def\PY@tc##1{\textcolor[rgb]{0.00,0.50,0.00}{##1}}}
\expandafter\def\csname PY@tok@nf\endcsname{\def\PY@tc##1{\textcolor[rgb]{0.00,0.00,1.00}{##1}}}
\expandafter\def\csname PY@tok@nc\endcsname{\let\PY@bf=\textbf\def\PY@tc##1{\textcolor[rgb]{0.00,0.00,1.00}{##1}}}
\expandafter\def\csname PY@tok@nn\endcsname{\let\PY@bf=\textbf\def\PY@tc##1{\textcolor[rgb]{0.00,0.00,1.00}{##1}}}
\expandafter\def\csname PY@tok@ne\endcsname{\let\PY@bf=\textbf\def\PY@tc##1{\textcolor[rgb]{0.82,0.25,0.23}{##1}}}
\expandafter\def\csname PY@tok@nv\endcsname{\def\PY@tc##1{\textcolor[rgb]{0.10,0.09,0.49}{##1}}}
\expandafter\def\csname PY@tok@no\endcsname{\def\PY@tc##1{\textcolor[rgb]{0.53,0.00,0.00}{##1}}}
\expandafter\def\csname PY@tok@nl\endcsname{\def\PY@tc##1{\textcolor[rgb]{0.63,0.63,0.00}{##1}}}
\expandafter\def\csname PY@tok@ni\endcsname{\let\PY@bf=\textbf\def\PY@tc##1{\textcolor[rgb]{0.60,0.60,0.60}{##1}}}
\expandafter\def\csname PY@tok@na\endcsname{\def\PY@tc##1{\textcolor[rgb]{0.49,0.56,0.16}{##1}}}
\expandafter\def\csname PY@tok@nt\endcsname{\let\PY@bf=\textbf\def\PY@tc##1{\textcolor[rgb]{0.00,0.50,0.00}{##1}}}
\expandafter\def\csname PY@tok@nd\endcsname{\def\PY@tc##1{\textcolor[rgb]{0.67,0.13,1.00}{##1}}}
\expandafter\def\csname PY@tok@s\endcsname{\def\PY@tc##1{\textcolor[rgb]{0.73,0.13,0.13}{##1}}}
\expandafter\def\csname PY@tok@sd\endcsname{\let\PY@it=\textit\def\PY@tc##1{\textcolor[rgb]{0.73,0.13,0.13}{##1}}}
\expandafter\def\csname PY@tok@si\endcsname{\let\PY@bf=\textbf\def\PY@tc##1{\textcolor[rgb]{0.73,0.40,0.53}{##1}}}
\expandafter\def\csname PY@tok@se\endcsname{\let\PY@bf=\textbf\def\PY@tc##1{\textcolor[rgb]{0.73,0.40,0.13}{##1}}}
\expandafter\def\csname PY@tok@sr\endcsname{\def\PY@tc##1{\textcolor[rgb]{0.73,0.40,0.53}{##1}}}
\expandafter\def\csname PY@tok@ss\endcsname{\def\PY@tc##1{\textcolor[rgb]{0.10,0.09,0.49}{##1}}}
\expandafter\def\csname PY@tok@sx\endcsname{\def\PY@tc##1{\textcolor[rgb]{0.00,0.50,0.00}{##1}}}
\expandafter\def\csname PY@tok@m\endcsname{\def\PY@tc##1{\textcolor[rgb]{0.40,0.40,0.40}{##1}}}
\expandafter\def\csname PY@tok@gh\endcsname{\let\PY@bf=\textbf\def\PY@tc##1{\textcolor[rgb]{0.00,0.00,0.50}{##1}}}
\expandafter\def\csname PY@tok@gu\endcsname{\let\PY@bf=\textbf\def\PY@tc##1{\textcolor[rgb]{0.50,0.00,0.50}{##1}}}
\expandafter\def\csname PY@tok@gd\endcsname{\def\PY@tc##1{\textcolor[rgb]{0.63,0.00,0.00}{##1}}}
\expandafter\def\csname PY@tok@gi\endcsname{\def\PY@tc##1{\textcolor[rgb]{0.00,0.63,0.00}{##1}}}
\expandafter\def\csname PY@tok@gr\endcsname{\def\PY@tc##1{\textcolor[rgb]{1.00,0.00,0.00}{##1}}}
\expandafter\def\csname PY@tok@ge\endcsname{\let\PY@it=\textit}
\expandafter\def\csname PY@tok@gs\endcsname{\let\PY@bf=\textbf}
\expandafter\def\csname PY@tok@gp\endcsname{\let\PY@bf=\textbf\def\PY@tc##1{\textcolor[rgb]{0.00,0.00,0.50}{##1}}}
\expandafter\def\csname PY@tok@go\endcsname{\def\PY@tc##1{\textcolor[rgb]{0.53,0.53,0.53}{##1}}}
\expandafter\def\csname PY@tok@gt\endcsname{\def\PY@tc##1{\textcolor[rgb]{0.00,0.27,0.87}{##1}}}
\expandafter\def\csname PY@tok@err\endcsname{\def\PY@bc##1{\setlength{\fboxsep}{0pt}\fcolorbox[rgb]{1.00,0.00,0.00}{1,1,1}{\strut ##1}}}
\expandafter\def\csname PY@tok@kc\endcsname{\let\PY@bf=\textbf\def\PY@tc##1{\textcolor[rgb]{0.00,0.50,0.00}{##1}}}
\expandafter\def\csname PY@tok@kd\endcsname{\let\PY@bf=\textbf\def\PY@tc##1{\textcolor[rgb]{0.00,0.50,0.00}{##1}}}
\expandafter\def\csname PY@tok@kn\endcsname{\let\PY@bf=\textbf\def\PY@tc##1{\textcolor[rgb]{0.00,0.50,0.00}{##1}}}
\expandafter\def\csname PY@tok@kr\endcsname{\let\PY@bf=\textbf\def\PY@tc##1{\textcolor[rgb]{0.00,0.50,0.00}{##1}}}
\expandafter\def\csname PY@tok@bp\endcsname{\def\PY@tc##1{\textcolor[rgb]{0.00,0.50,0.00}{##1}}}
\expandafter\def\csname PY@tok@fm\endcsname{\def\PY@tc##1{\textcolor[rgb]{0.00,0.00,1.00}{##1}}}
\expandafter\def\csname PY@tok@vc\endcsname{\def\PY@tc##1{\textcolor[rgb]{0.10,0.09,0.49}{##1}}}
\expandafter\def\csname PY@tok@vg\endcsname{\def\PY@tc##1{\textcolor[rgb]{0.10,0.09,0.49}{##1}}}
\expandafter\def\csname PY@tok@vi\endcsname{\def\PY@tc##1{\textcolor[rgb]{0.10,0.09,0.49}{##1}}}
\expandafter\def\csname PY@tok@vm\endcsname{\def\PY@tc##1{\textcolor[rgb]{0.10,0.09,0.49}{##1}}}
\expandafter\def\csname PY@tok@sa\endcsname{\def\PY@tc##1{\textcolor[rgb]{0.73,0.13,0.13}{##1}}}
\expandafter\def\csname PY@tok@sb\endcsname{\def\PY@tc##1{\textcolor[rgb]{0.73,0.13,0.13}{##1}}}
\expandafter\def\csname PY@tok@sc\endcsname{\def\PY@tc##1{\textcolor[rgb]{0.73,0.13,0.13}{##1}}}
\expandafter\def\csname PY@tok@dl\endcsname{\def\PY@tc##1{\textcolor[rgb]{0.73,0.13,0.13}{##1}}}
\expandafter\def\csname PY@tok@s2\endcsname{\def\PY@tc##1{\textcolor[rgb]{0.73,0.13,0.13}{##1}}}
\expandafter\def\csname PY@tok@sh\endcsname{\def\PY@tc##1{\textcolor[rgb]{0.73,0.13,0.13}{##1}}}
\expandafter\def\csname PY@tok@s1\endcsname{\def\PY@tc##1{\textcolor[rgb]{0.73,0.13,0.13}{##1}}}
\expandafter\def\csname PY@tok@mb\endcsname{\def\PY@tc##1{\textcolor[rgb]{0.40,0.40,0.40}{##1}}}
\expandafter\def\csname PY@tok@mf\endcsname{\def\PY@tc##1{\textcolor[rgb]{0.40,0.40,0.40}{##1}}}
\expandafter\def\csname PY@tok@mh\endcsname{\def\PY@tc##1{\textcolor[rgb]{0.40,0.40,0.40}{##1}}}
\expandafter\def\csname PY@tok@mi\endcsname{\def\PY@tc##1{\textcolor[rgb]{0.40,0.40,0.40}{##1}}}
\expandafter\def\csname PY@tok@il\endcsname{\def\PY@tc##1{\textcolor[rgb]{0.40,0.40,0.40}{##1}}}
\expandafter\def\csname PY@tok@mo\endcsname{\def\PY@tc##1{\textcolor[rgb]{0.40,0.40,0.40}{##1}}}
\expandafter\def\csname PY@tok@ch\endcsname{\let\PY@it=\textit\def\PY@tc##1{\textcolor[rgb]{0.25,0.50,0.50}{##1}}}
\expandafter\def\csname PY@tok@cm\endcsname{\let\PY@it=\textit\def\PY@tc##1{\textcolor[rgb]{0.25,0.50,0.50}{##1}}}
\expandafter\def\csname PY@tok@cpf\endcsname{\let\PY@it=\textit\def\PY@tc##1{\textcolor[rgb]{0.25,0.50,0.50}{##1}}}
\expandafter\def\csname PY@tok@c1\endcsname{\let\PY@it=\textit\def\PY@tc##1{\textcolor[rgb]{0.25,0.50,0.50}{##1}}}
\expandafter\def\csname PY@tok@cs\endcsname{\let\PY@it=\textit\def\PY@tc##1{\textcolor[rgb]{0.25,0.50,0.50}{##1}}}

\def\PYZbs{\char`\\}
\def\PYZus{\char`\_}
\def\PYZob{\char`\{}
\def\PYZcb{\char`\}}
\def\PYZca{\char`\^}
\def\PYZam{\char`\&}
\def\PYZlt{\char`\<}
\def\PYZgt{\char`\>}
\def\PYZsh{\char`\#}
\def\PYZpc{\char`\%}
\def\PYZdl{\char`\$}
\def\PYZhy{\char`\-}
\def\PYZsq{\char`\'}
\def\PYZdq{\char`\"}
\def\PYZti{\char`\~}
% for compatibility with earlier versions
\def\PYZat{@}
\def\PYZlb{[}
\def\PYZrb{]}
\makeatother


    % For linebreaks inside Verbatim environment from package fancyvrb. 
    \makeatletter
        \newbox\Wrappedcontinuationbox 
        \newbox\Wrappedvisiblespacebox 
        \newcommand*\Wrappedvisiblespace {\textcolor{red}{\textvisiblespace}} 
        \newcommand*\Wrappedcontinuationsymbol {\textcolor{red}{\llap{\tiny$\m@th\hookrightarrow$}}} 
        \newcommand*\Wrappedcontinuationindent {3ex } 
        \newcommand*\Wrappedafterbreak {\kern\Wrappedcontinuationindent\copy\Wrappedcontinuationbox} 
        % Take advantage of the already applied Pygments mark-up to insert 
        % potential linebreaks for TeX processing. 
        %        {, <, #, %, $, ' and ": go to next line. 
        %        _, }, ^, &, >, - and ~: stay at end of broken line. 
        % Use of \textquotesingle for straight quote. 
        \newcommand*\Wrappedbreaksatspecials {% 
            \def\PYGZus{\discretionary{\char`\_}{\Wrappedafterbreak}{\char`\_}}% 
            \def\PYGZob{\discretionary{}{\Wrappedafterbreak\char`\{}{\char`\{}}% 
            \def\PYGZcb{\discretionary{\char`\}}{\Wrappedafterbreak}{\char`\}}}% 
            \def\PYGZca{\discretionary{\char`\^}{\Wrappedafterbreak}{\char`\^}}% 
            \def\PYGZam{\discretionary{\char`\&}{\Wrappedafterbreak}{\char`\&}}% 
            \def\PYGZlt{\discretionary{}{\Wrappedafterbreak\char`\<}{\char`\<}}% 
            \def\PYGZgt{\discretionary{\char`\>}{\Wrappedafterbreak}{\char`\>}}% 
            \def\PYGZsh{\discretionary{}{\Wrappedafterbreak\char`\#}{\char`\#}}% 
            \def\PYGZpc{\discretionary{}{\Wrappedafterbreak\char`\%}{\char`\%}}% 
            \def\PYGZdl{\discretionary{}{\Wrappedafterbreak\char`\$}{\char`\$}}% 
            \def\PYGZhy{\discretionary{\char`\-}{\Wrappedafterbreak}{\char`\-}}% 
            \def\PYGZsq{\discretionary{}{\Wrappedafterbreak\textquotesingle}{\textquotesingle}}% 
            \def\PYGZdq{\discretionary{}{\Wrappedafterbreak\char`\"}{\char`\"}}% 
            \def\PYGZti{\discretionary{\char`\~}{\Wrappedafterbreak}{\char`\~}}% 
        } 
        % Some characters . , ; ? ! / are not pygmentized. 
        % This macro makes them "active" and they will insert potential linebreaks 
        \newcommand*\Wrappedbreaksatpunct {% 
            \lccode`\~`\.\lowercase{\def~}{\discretionary{\hbox{\char`\.}}{\Wrappedafterbreak}{\hbox{\char`\.}}}% 
            \lccode`\~`\,\lowercase{\def~}{\discretionary{\hbox{\char`\,}}{\Wrappedafterbreak}{\hbox{\char`\,}}}% 
            \lccode`\~`\;\lowercase{\def~}{\discretionary{\hbox{\char`\;}}{\Wrappedafterbreak}{\hbox{\char`\;}}}% 
            \lccode`\~`\:\lowercase{\def~}{\discretionary{\hbox{\char`\:}}{\Wrappedafterbreak}{\hbox{\char`\:}}}% 
            \lccode`\~`\?\lowercase{\def~}{\discretionary{\hbox{\char`\?}}{\Wrappedafterbreak}{\hbox{\char`\?}}}% 
            \lccode`\~`\!\lowercase{\def~}{\discretionary{\hbox{\char`\!}}{\Wrappedafterbreak}{\hbox{\char`\!}}}% 
            \lccode`\~`\/\lowercase{\def~}{\discretionary{\hbox{\char`\/}}{\Wrappedafterbreak}{\hbox{\char`\/}}}% 
            \catcode`\.\active
            \catcode`\,\active 
            \catcode`\;\active
            \catcode`\:\active
            \catcode`\?\active
            \catcode`\!\active
            \catcode`\/\active 
            \lccode`\~`\~ 	
        }
    \makeatother

    \let\OriginalVerbatim=\Verbatim
    \makeatletter
    \renewcommand{\Verbatim}[1][1]{%
        %\parskip\z@skip
        \sbox\Wrappedcontinuationbox {\Wrappedcontinuationsymbol}%
        \sbox\Wrappedvisiblespacebox {\FV@SetupFont\Wrappedvisiblespace}%
        \def\FancyVerbFormatLine ##1{\hsize\linewidth
            \vtop{\raggedright\hyphenpenalty\z@\exhyphenpenalty\z@
                \doublehyphendemerits\z@\finalhyphendemerits\z@
                \strut ##1\strut}%
        }%
        % If the linebreak is at a space, the latter will be displayed as visible
        % space at end of first line, and a continuation symbol starts next line.
        % Stretch/shrink are however usually zero for typewriter font.
        \def\FV@Space {%
            \nobreak\hskip\z@ plus\fontdimen3\font minus\fontdimen4\font
            \discretionary{\copy\Wrappedvisiblespacebox}{\Wrappedafterbreak}
            {\kern\fontdimen2\font}%
        }%
        
        % Allow breaks at special characters using \PYG... macros.
        \Wrappedbreaksatspecials
        % Breaks at punctuation characters . , ; ? ! and / need catcode=\active 	
        \OriginalVerbatim[#1,codes*=\Wrappedbreaksatpunct]%
    }
    \makeatother

    % Exact colors from NB
    \definecolor{incolor}{HTML}{303F9F}
    \definecolor{outcolor}{HTML}{D84315}
    \definecolor{cellborder}{HTML}{CFCFCF}
    \definecolor{cellbackground}{HTML}{F7F7F7}
    
    % prompt
    \makeatletter
    \newcommand{\boxspacing}{\kern\kvtcb@left@rule\kern\kvtcb@boxsep}
    \makeatother
    \newcommand{\prompt}[4]{
        \ttfamily\llap{{\color{#2}[#3]:\hspace{3pt}#4}}\vspace{-\baselineskip}
    }
    

    
    % Prevent overflowing lines due to hard-to-break entities
    \sloppy 
    % Setup hyperref package
    \hypersetup{
      breaklinks=true,  % so long urls are correctly broken across lines
      colorlinks=true,
      urlcolor=urlcolor,
      linkcolor=linkcolor,
      citecolor=citecolor,
      }
    % Slightly bigger margins than the latex defaults
    
    \geometry{verbose,tmargin=1in,bmargin=1in,lmargin=1in,rmargin=1in}
    
\begin{document}
   
%    \maketitle
%	ref: https://stackoverflow.com/questions/3141702/vertically-centering-a-title-page
	\begin{titlepage}
		\vspace*{\fill}
		\begin{center}
 		\normalsize Laporan Implementasi \\
		\huge Prediksi Debit Aliran menggunakan \emph{Long Short-Term Memory} (LSTM)\\ 
		\normalsize Versi (Rapih) 1.0.0 \\[0.2cm]
      	\small Berdasarkan \emph{Jupyter Notebook}: \texttt{github\_taruma\_demo\_lstm\_rr.ipynb} \\[0.5cm]
		\normalsize oleh Taruma Sakti Megariansyah\\[0.5cm]
      	\normalsize 22 Oktober 2019\\[1cm]
    	\adjustimage{max size={0.9\linewidth}{1cm}}{github_taruma_demo_lstm_rr_files/vivaldi_logo.png}\\
      	\normalsize github.com/taruma/vivaldi
		\end{center}
    	\vspace*{\fill}
	\end{titlepage}
    

    
    \hypertarget{prediksi-debit-aliran-menggunakan-long-short-term-memory-lstm}{%
\section{\texorpdfstring{Prediksi Debit Aliran Menggunakan \emph{Long
Short-Term Memory}
(LSTM)}{Prediksi Debit Aliran Menggunakan Long Short-Term Memory (LSTM)}}\label{prediksi-debit-aliran-menggunakan-long-short-term-memory-lstm}}

\emph{Jupyter Notebook} (selanjutnya disebut buku) ini hanya
\textbf{contoh} dan dibuat untuk \textbf{pembelajaran} mengenai
\emph{Deep Learning/Neural Networks} dan mendemonstrasikan penggunaan
\emph{Python} di bidang sumberdaya air. Buku ini masih perlu dievaluasi
kembali jika digunakan untuk kepentingan riset/penelitian ataupun
proyek.

Buku ini disertai catatan yang berisikan penjelasan lebih lanjut
mengenai buku ini (daftar pustaka, penjelasan dataset, dll). Catatan
dapat diunduh di bagian unduh buku.

\hypertarget{pranala-buku}{%
\subsection{Pranala buku}\label{pranala-buku}}

Buku ini bisa diunduh dengan berbagai format. Versi Google Colab akan
lebih diperbarui dibandingkan versi lainnya.

\begin{itemize}
\tightlist
\item
  \href{https://colab.research.google.com/drive/1bx3ak_20dcJ7VdGR-djysLIxLaX7pRI2}{Pranala
  Google Colab}, format Google Colab versi terakhir
\item
  \href{https://github.com/taruma/vivaldi/blob/master/notebook/github_taruma_demo_lstm_rr.ipynb}{Pranala
  Github}, format .ipynb versi 1.0.0
\item
  \href{https://nbviewer.jupyter.org/github/taruma/vivaldi/blob/master/notebook/github_taruma_demo_lstm_rr.ipynb}{Lihat
  melalui NBViewer}, format .ipynb versi Github
\item
  \href{https://github.com/taruma/vivaldi/blob/master/pdf/taruma_lstm_rr_laporan.pdf?raw=true}{Unduh
  Laporan}, format PDF versi 1.0.0 dengan \emph{source code} +
  \emph{outputs}
\item
  \href{https://github.com/taruma/vivaldi/blob/master/pdf/taruma_lstm_rr_laporan_rapih.pdf?raw=true}{Unduh
  Laporan (rapih)}, format PDF versi 1.0.0 hanya \emph{outputs}*
\item
  \href{https://github.com/taruma/vivaldi/blob/master/pdf/taruma_lstm_rr_catatan.pdf?raw=true}{Unduh
  Catatan}, format PDF versi 1.0.0
\end{itemize}

Pembuatan laporan dilakukan dengan mengubah buku ke dalam bentuk LaTeX
dan dilakukan perubahan sedikit, sehingga disarankan untuk mengunduh
versi laporan (rapih).

\hypertarget{catatan}{%
\subsection{Catatan}\label{catatan}}

\begin{itemize}
\tightlist
\item
  Buku ini dikembangkan menggunakan
  \href{https://colab.research.google.com/}{Google Colab}, sehingga
  penampilan terbaik dan interaktif dari buku ini diperoleh jika dibuka
  melalui Google Colab.
\item
  Anda dapat mendiskusikan mengenai buku ini (atau hal lainnya seperti
  koreksi, kritik, saran, pertanyaan, dll) melalui isu di
  \emph{repository}
  \href{https://github.com/taruma/vivaldi}{taruma/vivaldi} atau dapat
  menghubungi saya melalui email \(hi@taruma.info\).
\item
  Buku ini masih perlu dievaluasi baik dari teori ataupun implementasi.
  Ini merupakan buku pribadi yang digunakan oleh saya sebagai latihan
  implementasi \emph{Deep Learning} menggunakan \emph{Python}. Referensi
  materi pembelajaran saya dapat dilihat pada catatan buku.
\item
  \textbf{Biasakan untuk selalu memeriksa kode terlebih dahulu sebelum
  menjalankannya untuk masalah keamanan}.
\end{itemize}

    \hypertarget{deskripsi-kasus}{%
\section{Deskripsi Kasus}\label{deskripsi-kasus}}

Bagian ini menjelaskan gambaran umum mengenai dataset,
permasalahan/tujuan, dan strategi penyelesaiannya.

    \hypertarget{dataset}{%
\subsection{Dataset}\label{dataset}}

Dataset merupakan data hidrologi dan klimatologi \textbf{harian} dari
tanggal \textbf{1 Januari 1998} sampai \textbf{31 Desember 2008} Daerah
Aliran Sungai (DAS) Bendung Baru Pamarayan. Dataset terpisah menjadi 3
kategori yaitu: data curah hujan, data klimatologi, dan data debit.

\begin{itemize}
\tightlist
\item
  Data curah hujan diperoleh dari 8 stasiun yaitu:
  \texttt{bojong\_manik}, \texttt{gunung\_tunggal}, \texttt{pasir\_ona},
  \texttt{sampang\_peundeuy}, \texttt{cimarga}, \texttt{bd\_pamarayan},
  \texttt{ciminyak\_cilaki}, \texttt{gardu\_tanjak}.
\item
  Data debit diperoleh dari 1 stasiun yaitu: \texttt{bd\_pamarayan}.
\item
  Data klimatologi diperoleh dari 1 stasiun yaitu:
  \texttt{geofisika\_serang}.
\end{itemize}

Rincian mengenai dataset bisa dibaca di catatan buku.

    \hypertarget{objektif}{%
\subsection{Objektif}\label{objektif}}

\hypertarget{tujuan}{%
\subsubsection{Tujuan}\label{tujuan}}

\begin{itemize}
\tightlist
\item
  Peneliti ingin mengetahui nilai debit berdasarkan data hidrologi dan
  klimatologi yang tersedia pada waktu sebelumnya.
\end{itemize}

\hypertarget{batasan-masalah}{%
\subsubsection{Batasan Masalah}\label{batasan-masalah}}

\begin{itemize}
\tightlist
\item
  Arsitektur (sel) \emph{Recurrent Neural Networks} yang akan digunakan
  adalah \emph{Long Short-Term Memory} (LSTM).
\item
  Data yang hilang (\texttt{NaN}) diisi menggunakan metode interpolasi
  linear.
\item
  Diasumsikan bahwa data tidak perlu diverifikasi.
\item
  Jika data yang hilang lebih dari 1 tahun berurutan, maka paramater
  tersebut akan diabaikan.
\item
  Pelatihan model (training) menggunakan data dari tahun \textbf{1998 -
  2006}.
\item
  Tidak dilakukan \emph{feature engineering}, kolom yang bertipe ordinal
  atau kategori diabaikan.
\item
  Tidak ada tahapan pemilihan model terbaik (\emph{model selection}).
  Parameter akan sembarang mengikuti tulisan Kratzert et al (2018).
\item
  Berdasarkan Kratzert et al (2018), pada buku ini mengikuti bahwa
  dataset hanya dibagi dua bagian yaitu \emph{train set} dan \emph{test
  set}, dimana validasi menggunakan \emph{test set}.
\end{itemize}

\hypertarget{pertanyaan}{%
\subsubsection{Pertanyaan}\label{pertanyaan}}

\begin{itemize}
\tightlist
\item
  Berapa nilai debit pada waktu \(t\) jika telah diketahui nilai
  observasi pada waktu \(timesteps\) hari sebelumnya?
\end{itemize}

    \hypertarget{tahap-0-pengaturan-awal-dan-inisiasi}{%
\section{TAHAP 0: Pengaturan Awal dan
Inisiasi}\label{tahap-0-pengaturan-awal-dan-inisiasi}}

Pada tahap ini akan dilakukan pengaturan awal dan inisiasi dengan
melakukan atau menjawab daftar berikut:

\begin{itemize}
\tightlist
\item
  Menentukan penggunaan \emph{runtime} lokal atau \emph{Google Colab}.
\item
  Menentukan nama buku/proyek dan versi (digunakan jika melakukan
  penyimpanan).
\item
  Memeriksa paket hidrokit.
\item
  Menampilkan versi paket yang digunakan pada sistem.
\item
  Impor paket utama yang akan digunakan (numpy, pandas, matplotlib).
\end{itemize}

    \hypertarget{pengaturan-buku}{%
\subsection{Pengaturan Buku}\label{pengaturan-buku}}

    \begin{Verbatim}[commandchars=\\\{\}]
:: INFORMASI RUNTIME
:: BUKU INI MENGGUNAKAN RUNTIME: GOOGLE COLAB
    \end{Verbatim}

    \begin{Verbatim}[commandchars=\\\{\}]
:: INFORMASI PROYEK/BUKU
:: [project\_title]: 20191022\_0807\_taruma\_demo\_lstm\_rr\_1\_0\_0
    \end{Verbatim}

    \begin{Verbatim}[commandchars=\\\{\}]
:: MENGGUNAKAN TENSORFLOW 2.x (GOOGLE COLAB)
TensorFlow 2.x selected.
    \end{Verbatim}

    \begin{Verbatim}[commandchars=\\\{\}]
:: MEMERIKSA PAKET HIDROKIT
:: INSTALASI PAKET HIDROKIT
  Building wheel for hidrokit (setup.py) {\ldots} done
    \end{Verbatim}

    \begin{Verbatim}[commandchars=\\\{\}]
:: INFORMASI VERSI SISTEM
::       python version: 3.6.8
::        numpy version: 1.16.5
::       pandas version: 0.24.2
::   matplotlib version: 3.0.3
::   tensorflow version: 2.0.0
::        keras version: 2.2.4-tf
::     hidrokit version: 0.3.2
    \end{Verbatim}

    \begin{Verbatim}[commandchars=\\\{\}]
:: LOKASI PENYIMPANAN DATASET
DATASET\_PATH = /content/gdrive/My Drive/Colab Notebooks/\_dataset/uma\_pamarayan
DROP\_PATH = /content/gdrive/My Drive/Colab Notebooks/\_dropbox
    \end{Verbatim}

    \hypertarget{persiapan-sistem}{%
\subsection{Persiapan sistem}\label{persiapan-sistem}}

    \begin{Verbatim}[commandchars=\\\{\}]
:: IMPORT LIBRARY (NUMPY, PANDAS, MATPLOTLIB)
    \end{Verbatim}

    \hypertarget{tahap-1-akuisisi-dataset}{%
\section{TAHAP 1: AKUISISI DATASET}\label{tahap-1-akuisisi-dataset}}

Pada tahap ini, tujuan utamanya adalah membaca seluruh dataset yang
dimiliki dan mengimpor dataset tersebut untuk pengolahan prapemrosesan
data (\emph{data preprocessing}).

    \hypertarget{data-hujan}{%
\subsection{DATA HUJAN}\label{data-hujan}}

Pada kasus ini, terdapat 8 stasiun yang akan digunakan sehingga terdapat
8 berkas excel. Setiap berkas memiliki data curah hujan dari tahun 1998
hingga 2008 yang disimpan pada masing-masing \emph{sheet} untuk setiap
tahunnya.

Untuk memperoleh data tersebut dalam bentuk tabel (bukan dalam bentuk
pivot) digunakan modul yang tersedia di \texttt{hidrokit} (hanya pada
versi 0.3.x). Modul dapat diakses melalui
\texttt{hidrokit.contrib.taruma.hk43}
(\href{https://nbviewer.jupyter.org/gist/taruma/a9dd4ea61db2526853b99600909e9c50}{panduan}).

    \begin{Verbatim}[commandchars=\\\{\}]
:: MEMBACA DATA HUJAN DARI [DATASET\_PATH]
Found 8 file(s)
::  1  :        hujan\_bojong\_manik\_1998\_2008.xls
::  2  :        hujan\_gunung\_tunggal\_1998\_2008.xls
::  3  :        hujan\_pasir\_ona\_1998\_2008.xls
::  4  :        hujan\_sampang\_peundeuy\_1998\_2008.xls
::  5  :        hujan\_cimarga\_1998\_2008.xls
::  6  :        hujan\_bd\_pamarayan\_1998\_2008.xls
::  7  :        hujan\_ciminyak\_cilaki\_1998\_2008.xls
::  8  :        hujan\_gardu\_tanjak\_1998\_2008.xls
    \end{Verbatim}

    \begin{Verbatim}[commandchars=\\\{\}]
:: tipe [hujan\_raw] = <class 'dict'>
:: tipe [hujan\_invalid] = <class 'dict'>
    \end{Verbatim}

    \hypertarget{data-debit}{%
\subsection{DATA DEBIT}\label{data-debit}}

Untuk data debit, dilakukan hal yang serupa dengan data hujan.

    \begin{Verbatim}[commandchars=\\\{\}]
:: MEMBACA DATA DEBIT DARI [DATASET\_PATH]
Found 1 file(s)
::  1  :        debit\_bd\_pamarayan\_1998\_2008.xls
    \end{Verbatim}

    \begin{Verbatim}[commandchars=\\\{\}]
:: tipe [debit\_raw] = <class 'dict'>
:: tipe [debit\_invalid] = <class 'dict'>
    \end{Verbatim}

    \hypertarget{data-klimatologi}{%
\subsection{DATA KLIMATOLOGI}\label{data-klimatologi}}

Data klimatologi diperoleh dari situs data online bmkg. Data klimatologi
dari bmkg lebih mudah di impor dikarenakan data sudah tersedia dalam
bentuk tabel.

Digunakan modul \texttt{hidrokit.contrib.taruma.hk73}
(\href{https://nbviewer.jupyter.org/gist/taruma/b00880905f297013f046dad95dc2e284}{panduan})
agar memudahkan proses impornya.

    \begin{Verbatim}[commandchars=\\\{\}]
:: [KLIMATOLOGI\_PATH] = /content/gdrive/My Drive/Colab
Notebooks/\_dataset/uma\_pamarayan/klimatologi\_serang\_1998\_2008.xlsx
:: MEMBACA DATA KLIMATOLOGI DARI [KLIMATOLOGI\_PATH]
:: tipe [df\_klimatologi] = <class 'pandas.core.frame.DataFrame'>
    \end{Verbatim}

    \hypertarget{tahap-2-prapemrosesan-data}{%
\section{TAHAP 2: PRAPEMROSESAN DATA}\label{tahap-2-prapemrosesan-data}}

Pada tahap ini, dataset yang telah diimpor akan diperiksa dan
dipersiapkan untuk pengolahan data di tahap selanjutnya. Berikut yang
dilakukan pada tahap ini:

\begin{itemize}
\tightlist
\item
  Memastikan dataset berupa \texttt{pandas.DataFrame}.
\item
  Memeriksa data yang invalid (salah input/bukan bilangan).
\item
  Mengoreksi nilai yang invalid.
\item
  Mengubah tipe data pada dataframe menjadi numerik.
\item
  Memeriksa data yang hilang (\texttt{NaN}).
\item
  Mengisi nilai hilang dengan metode interpolasi linear.
\item
  Menyesuaikan kelengkapan dataset.
\end{itemize}

    \hypertarget{data-hujan}{%
\subsection{DATA HUJAN}\label{data-hujan}}

    \begin{Verbatim}[commandchars=\\\{\}]
:: MENGUBAH [hujan\_raw] MENJADI [df\_hujan] SEBAGAI DATAFRAME
:: MENAMPILKAN [df\_hujan]
    \end{Verbatim}

            \begin{tcolorbox}[breakable, size=fbox, boxrule=.5pt, pad at break*=1mm, opacityfill=0]
\prompt{Out}{outcolor}{0}{\boxspacing}
\begin{Verbatim}[commandchars=\\\{\}]
           hujan\_bojong\_manik  {\ldots} hujan\_gardu\_tanjak
1998-01-01                  -  {\ldots}                  -
1998-01-02                  -  {\ldots}                  -
1998-01-03                  5  {\ldots}                  5
1998-01-04                  -  {\ldots}                  -
1998-01-05                  -  {\ldots}                  -

[5 rows x 8 columns]
\end{Verbatim}
\end{tcolorbox}
        
    \begin{Verbatim}[commandchars=\\\{\}]
:: MEMERIKSA NILAI INVALID PADA [df\_hujan]
:: MENAMPILKAN NILAI INVALID
:: NILAI INVALID BERUPA = ['-', 'NaN']
    \end{Verbatim}

    Dari \texttt{hujan\_invalid} diketahui bahwa pada data hujan memiliki
data invalid berupa isian ``\texttt{-}'' dan ``\texttt{NaN}''. Isian
``\texttt{-}'' pada data curah hujan menandakan bahwa tidak ada hujan
atau bernilai \texttt{0.}. Sedangkan data ``\texttt{NaN}'' menandakan
bahwa data tidak terekam sama sekali sehingga tidak diketahui terjadi
hujan atau tidak.

    \begin{Verbatim}[commandchars=\\\{\}]
:: MENGOREKSI NILAI INVALID PADA [df\_hujan]
:: MENGOREKSI NILAI "-" MENJADI 0.0
    \end{Verbatim}

    \begin{Verbatim}[commandchars=\\\{\}]
:: MENGUBAH TIPE DATA PADA DATAFRAME [df\_hujan]
:: MENAMPILKAN INFORMASI DATAFRAME [df\_hujan]:
<class 'pandas.core.frame.DataFrame'>
DatetimeIndex: 4018 entries, 1998-01-01 to 2008-12-31
Freq: D
Data columns (total 8 columns):
hujan\_bojong\_manik        4017 non-null float64
hujan\_gunung\_tunggal      4018 non-null float64
hujan\_pasir\_ona           4018 non-null float64
hujan\_sampang\_peundeuy    4016 non-null float64
hujan\_cimarga             4018 non-null float64
hujan\_bd\_pamarayan        4016 non-null float64
hujan\_ciminyak\_cilaki     4018 non-null float64
hujan\_gardu\_tanjak        4018 non-null float64
dtypes: float64(8)
memory usage: 282.5 KB
    \end{Verbatim}

    Dari informasi diatas diketahui bahwa tipe data pada dataframe telah
diubah menjadi berbentuk numerik.

    \begin{Verbatim}[commandchars=\\\{\}]
:: MENGISI NILAI HILANG MENGGUNAKAN METODE INTERPOLASI LINEAR
    \end{Verbatim}

    \begin{Verbatim}[commandchars=\\\{\}]
:: MEMERIKSA JIKA [df\_hujan] MASIH MEMILIKI NILAI YANG HILANG: False
    \end{Verbatim}

    \hypertarget{data-debit}{%
\subsection{DATA DEBIT}\label{data-debit}}

Langkahnya serupa dengan data hujan.

    \begin{Verbatim}[commandchars=\\\{\}]
:: MENGUBAH [debit\_raw] MENJADI [df\_debit] SEBAGAI DATAFRAME
:: MENAMPILKAN [df\_debit]
    \end{Verbatim}

            \begin{tcolorbox}[breakable, size=fbox, boxrule=.5pt, pad at break*=1mm, opacityfill=0]
\prompt{Out}{outcolor}{0}{\boxspacing}
\begin{Verbatim}[commandchars=\\\{\}]
           debit\_bd\_pamarayan
1998-01-01                  0
1998-01-02                  0
1998-01-03                  0
1998-01-04                  0
1998-01-05                  0
\end{Verbatim}
\end{tcolorbox}
        
    \begin{Verbatim}[commandchars=\\\{\}]
:: MEMERIKSA NILAI INVALID PADA [df\_debit]
:: MENAMPILKAN NILAI INVALID
:: NILAI INVALID BERUPA = ['20.9.46', 'NaN', 'tad']
    \end{Verbatim}

    Dari \texttt{debit\_invalid} diketahui bahwa \texttt{df\_debit} memiliki
nilai invalid berupa ``\texttt{20.9.46}'', ``\texttt{NaN}'', dan
``\texttt{tad}''. Nilai ``\texttt{tad}'' diartikan sebagai tidak ada
data, sedangkan ``\texttt{NaN}'' adalah data yang hilang. Untuk nilai
``\texttt{20.9.46}'', diasumsikan terjadi kekeliruan saat memasukkan
nilai, nilai tersebut dikoreksi menjadi \texttt{209.46}.

    \begin{Verbatim}[commandchars=\\\{\}]
:: MENGOREKSI NILAI 20.9.46 MENJADI 209.46
:: MENGOREKSI NILAI tad MENJADI NaN
    \end{Verbatim}

    \begin{Verbatim}[commandchars=\\\{\}]
:: MENGUBAH TIPE DATA PADA DATAFRAME [df\_debit]
:: MENAMPILKAN INFORMASI DATAFRAME [df\_debit]:
<class 'pandas.core.frame.DataFrame'>
DatetimeIndex: 4018 entries, 1998-01-01 to 2008-12-31
Freq: D
Data columns (total 1 columns):
debit\_bd\_pamarayan    4016 non-null float64
dtypes: float64(1)
memory usage: 62.8 KB
    \end{Verbatim}

    Dari informasi diatas diketahui bahwa tipe data pada dataframe telah
diubah menjadi berbentuk numerik.

    \begin{Verbatim}[commandchars=\\\{\}]
:: MENGISI NILAI HILANG MENGGUNAKAN METODE INTERPOLASI LINEAR
    \end{Verbatim}

    \begin{Verbatim}[commandchars=\\\{\}]
:: MEMERIKSA JIKA [df\_debit] MASIH MEMILIKI NILAI YANG HILANG: False
    \end{Verbatim}

    Nilai \texttt{0.} dapat berarti terjadi kekeringan atau pengeringan
(pada berkas tercantum ``pengeringan'' pada periode 13 Oktober 2000-31
Oktober 2000).

    \begin{Verbatim}[commandchars=\\\{\}]
:: MEMERIKSA KEKERINGAN PADA DATA DEBIT
:: Kekeringan terjadi pada tanggal: ['01 Jan 1998-28 Feb 1998', '15 Mar 1999',
'29 Oct 1999-31 Oct 1999', '13 Oct 2001-31 Oct 2001', '24 Oct 2003-25 Oct 2003',
'15 Jun 2004', '31 Aug 2004', '16 Nov 2004-30 Nov 2004', '08 Oct 2005', '11 Oct
2005-13 Oct 2005', '02 Oct 2006-19 Oct 2006', '16 Oct 2007-18 Oct 2007', '07 Sep
2008', '17 Oct 2008-18 Oct 2008']
    \end{Verbatim}

    Pada awal dataset (1 Januari 1998-28 Februari 1998) selama dua bulan
bernilai 0. secara beruntun. Saya mengasumsikan ini bukan terkait
kekeringan/pengeringan, melainkan data pengukuran dimulai pada bulan
maret tahun 1998. Sehingga, saya simpulkan bahwa dalam pemodelan
\textbf{dataset akan dimulai pada tanggal 1 Maret 1998}.

    \hypertarget{data-klimatologi}{%
\subsection{DATA KLIMATOLOGI}\label{data-klimatologi}}

Proses pada data klimatologi tidak jauh berbeda dengan proses data
hujan/debit. Akan tetapi karena data klimatologi berasalkan dari sumber
berbeda (BMKG), maka implementasinya akan berbeda dengan implementasi
pada data hujan/debit.

Pada modul \texttt{hidrokit.contrib.taruma.hk73}
(\href{https://nbviewer.jupyter.org/gist/taruma/b00880905f297013f046dad95dc2e284}{panduan})
telah disiapkan beberapa fungsi yang memudahkan untuk memeriksa data
klimatologi.

    \hypertarget{persiapan}{%
\subsubsection{Persiapan}\label{persiapan}}

Pada tahap 1, data klimatologi telah berbentuk \texttt{DataFrame},
sehingga dapat langsung dilakukan prapemrosesan data.

    \begin{Verbatim}[commandchars=\\\{\}]
:: MENAMPILKAN DATAFRAME [df\_klimatologi]:
    \end{Verbatim}

            \begin{tcolorbox}[breakable, size=fbox, boxrule=.5pt, pad at break*=1mm, opacityfill=0]
\prompt{Out}{outcolor}{0}{\boxspacing}
\begin{Verbatim}[commandchars=\\\{\}]
              Tn    Tx  Tavg  RH\_avg   RR   ss  ff\_x  ddd\_x  ff\_avg ddd\_car
Tanggal
1998-01-01  23.0  34.6  27.5      75  0.0  5.8     5    225       2      SW
1998-01-02  23.2  34.2  28.6      69  0.0  7.6     4    270       1      NE
1998-01-03  24.0  34.6  27.7      76  0.0  5.6     7    270       2      W
1998-01-04  23.8  34.4  28.4      70  0.0  8.0     7    225       3      SW
1998-01-05  23.5  33.4  27.7      74  1.0  3.5     6    270       2      W
\end{Verbatim}
\end{tcolorbox}
        
    Sebelum melanjutkan dalam prapemrosesan data pada data klimatologi,
terdapat beberapa kolom yang dihilangkan karena batasan masalah buku ini
dan mempermudah saat pemodelan. Kolom yang digunakan hanya kolom numerik
yang bersifat kontinu.\\
Berikut kolom yang dihilangkan:

\begin{itemize}
\tightlist
\item
  kolom \texttt{ddd\_car}, kolom ini merupakan kolom kategori yang tidak
  berupa angka.
\item
  kolom \texttt{ff\_x}, \texttt{ddd\_x}, \texttt{ff\_avg}, kolom ini
  (dapat) berupa kolom ordinal.
\end{itemize}

Kolom tersebut dapat diubah melalui proses \emph{feature engineering}.
Referensi lanjut bisa baca
\href{https://towardsdatascience.com/smarter-ways-to-encode-categorical-data-for-machine-learning-part-1-of-3-6dca2f71b159}{di
sini} dan
\href{https://towardsdatascience.com/basic-feature-engineering-to-reach-more-efficient-machine-learning-6294022e17a5}{di
sini}.

Selain itu, berdasarkan Megariansyah (2015) kolom \texttt{RR} (curah
hujan) tidak dapat digunakan karena stasiun tersebut tidak termasuk pada
wilayah Daerah Aliran Sungai (DAS) yang dikaji.

    \begin{Verbatim}[commandchars=\\\{\}]
:: MEMBERSIHKAN [df\_klimatologi] KE DATAFRAME [df\_klimatologi\_clean]
    \end{Verbatim}

            \begin{tcolorbox}[breakable, size=fbox, boxrule=.5pt, pad at break*=1mm, opacityfill=0]
\prompt{Out}{outcolor}{0}{\boxspacing}
\begin{Verbatim}[commandchars=\\\{\}]
              Tn    Tx  Tavg  RH\_avg   ss
Tanggal
1998-01-01  23.0  34.6  27.5      75  5.8
1998-01-02  23.2  34.2  28.6      69  7.6
1998-01-03  24.0  34.6  27.7      76  5.6
1998-01-04  23.8  34.4  28.4      70  8.0
1998-01-05  23.5  33.4  27.7      74  3.5
\end{Verbatim}
\end{tcolorbox}
        
    \hypertarget{prapemrosesan}{%
\subsubsection{Prapemrosesan}\label{prapemrosesan}}

Dilanjutkan dengan tahap prapemrosesan seperti memeriksa data invalid
ataupun kehilangan data.

    \begin{Verbatim}[commandchars=\\\{\}]
:: MEMERIKSA DATA YANG HILANG PADA [df\_klimatologi\_clean]
:: [df\_klimatologi\_clean] memiliki kehilangan data: True
:: Kolom yang memiliki data hilang: ['ss']
    \end{Verbatim}

    Diketahui bahwa pada kolom \texttt{ss} terdapat kehilangan data
``\texttt{NaN}''. Cek apakah kehilangan data terjadi secara beruntun.

    \begin{Verbatim}[commandchars=\\\{\}]
:: MEMERIKSA KONDISI DATA HILANG PADA [df\_klimatologi\_clean].ss
:: Tanggal terjadinya kehilangan data: ['30 Apr 2003']
    \end{Verbatim}

    Karena kehilangan data hanya terjadi pada satu hari, maka kolom
\texttt{ss} akan digunakan dalam pemodelan.

    Berdasarkan situs BMKG, harus diperiksa juga mengenai nilai
\texttt{8888} dan \texttt{9999} yang menandakan bahwa data tidak terukur
dan/atau tidak ada data. Data tersebut akan dikoreksi menjadi nilai
hilang (\texttt{NaN}) dan akan diisi menggunakan metode interpolasi.

    \begin{Verbatim}[commandchars=\\\{\}]
:: MEMERIKSA DATA YANG TIDAK ADA/TEREKAM PADA [df\_klimatologi\_clean]
\{'Tn': array([], dtype=int64), 'Tx': array([], dtype=int64), 'Tavg': array([],
dtype=int64), 'RH\_avg': array([], dtype=int64), 'ss': array([], dtype=int64)\}
    \end{Verbatim}

    Ternyata, pada kolom lain tidak memiliki nilai yang tidak terukur/tidak
ada. Sehingga, langkah berikutnya mengisi nilai hilang menggunakan
metode interpolasi linear.

    \begin{Verbatim}[commandchars=\\\{\}]
:: MENGISI NILAI YANG HILANG MENGGUNAKAN METODE INTERPOLASI LINEAR
    \end{Verbatim}

    \begin{Verbatim}[commandchars=\\\{\}]
:: MEMERIKSA JIKA [df\_klimatologi\_clean] MASIH MEMILIKI NILAI YANG HILANG: False
    \end{Verbatim}

    \hypertarget{penggabungan-dataset}{%
\subsection{PENGGABUNGAN DATASET}\label{penggabungan-dataset}}

Ketiga data yaitu data hujan, data klimatologi, dan data debit
digabungkan dalam satu DataFrame untuk pemodelan. Data gabungan sudah
dipastikan tidak memiliki nilai yang invalid atau data yang hilang.

Berdasarkan prapemrosesan data debit, diketahui bahwa dua bulan pertama
(Januari-Februari 1998) tidak memiliki data, maka dataset akan
menggunakan periode yang dimulai dari 1 Maret 1998.

    \begin{Verbatim}[commandchars=\\\{\}]
:: MENENTUKAN PERIODE DATASET
:: PERIODE DATASET DARI 19980301 hingga 20081231
:: DataFrame [data\_hujan], [data\_debit], [data\_klimatologi]
    \end{Verbatim}

    \begin{Verbatim}[commandchars=\\\{\}]
:: MENGGABUNGKAN SELURUH DATA DALAM SATU DATAFRAME [dataset]
    \end{Verbatim}

    \begin{Verbatim}[commandchars=\\\{\}]
:: MENAMAI ULANG NAMA KOLOM
:: INFO [dataset]:
<class 'pandas.core.frame.DataFrame'>
DatetimeIndex: 3959 entries, 1998-03-01 to 2008-12-31
Freq: D
Data columns (total 14 columns):
ch\_A               3959 non-null float64
ch\_B               3959 non-null float64
ch\_C               3959 non-null float64
ch\_D               3959 non-null float64
ch\_E               3959 non-null float64
ch\_F               3959 non-null float64
ch\_G               3959 non-null float64
ch\_H               3959 non-null float64
suhu\_min           3959 non-null float64
suhu\_max           3959 non-null float64
suhu\_rerata        3959 non-null float64
lembab\_rerata      3959 non-null int64
lama\_penyinaran    3959 non-null float64
debit              3959 non-null float64
dtypes: float64(13), int64(1)
memory usage: 623.9 KB
    \end{Verbatim}

    \begin{Verbatim}[commandchars=\\\{\}]
:: STATISTIK DESKRIPTIF [dataset]
    \end{Verbatim}

            \begin{tcolorbox}[breakable, size=fbox, boxrule=.5pt, pad at break*=1mm, opacityfill=0]
\prompt{Out}{outcolor}{0}{\boxspacing}
\begin{Verbatim}[commandchars=\\\{\}]
                      mean         std   min    50\%      max
ch\_A              6.222417   12.044965   0.0   0.00   180.00
ch\_B              6.087017   12.304554   0.0   0.00    99.50
ch\_C              6.249937   13.616252   0.0   0.00   135.00
ch\_D              6.949987   14.510073   0.0   0.00   140.00
ch\_E              5.732609   13.728183   0.0   0.00   133.00
ch\_F              5.316241   13.329691   0.0   0.00   163.00
ch\_G              7.955418   17.472431   0.0   0.00   275.00
ch\_H              8.469437   14.768919   0.0   0.00   148.00
suhu\_min         23.139227    1.026656  17.4  23.20    26.00
suhu\_max         31.826421    1.376400  25.6  32.00    36.20
suhu\_rerata      26.734731    0.844836  23.5  26.80    30.10
lembab\_rerata    81.678707    5.295782  59.0  82.00    97.00
lama\_penyinaran   4.840162    2.721498   0.0   5.20     8.00
debit            72.137286  105.077023   0.0  24.08  2561.58
\end{Verbatim}
\end{tcolorbox}
        
    \hypertarget{visualisasi}{%
\subsubsection{Visualisasi}\label{visualisasi}}

    \begin{center}
    \adjustimage{max size={0.9\linewidth}{0.9\paperheight}}{github_taruma_demo_lstm_rr_files/github_taruma_demo_lstm_rr_67_0.png}
    \end{center}
    { \hspace*{\fill} \\}
    
    \begin{center}
    \adjustimage{max size={0.9\linewidth}{0.9\paperheight}}{github_taruma_demo_lstm_rr_files/github_taruma_demo_lstm_rr_68_0.png}
    \end{center}
    { \hspace*{\fill} \\}
    
    Dari grafik diatas dapat beberapa informasi baru yang diperoleh berupa:

\begin{itemize}
\tightlist
\item
  Terlihat nilai luaran \emph{outlier} pada kolom debit yang terjadi di
  sekitar tahun 2001.
\item
  Untuk data hujan, hanya pada stasiun G yang memiliki rentang nilai
  dari 0-200, sedangkan yang lain bernilai 0-100.
\end{itemize}

Nilai luaran akan disertakan dalam pemodelan. Dan pada akhir tahap ini,
diasumsikan bahwa dataset sudah terverifikasi dan tervalidasi disertai
selesai melewati prapemrosesan data. Sehingga, pada tahap berikutnya,
dataset akan dianggap sudah memenuhi kriteria untuk pemodelan.

    \hypertarget{tahap-3-input-pemodelan}{%
\section{TAHAP 3: INPUT PEMODELAN}\label{tahap-3-input-pemodelan}}

Pada tahap ini akan fokus dalam persiapan input pemodelan. Langkah yang
akan dilakukan antara lain:

\begin{enumerate}
\def\labelenumi{\arabic{enumi}.}
\tightlist
\item
  Membagi dataset menjadi dua bagian yaitu \emph{train set} dan
  \emph{test set}.
\item
  Normalisasi/Standarisasi nilai pada dataset.
\item
  Mempersiapkan \emph{input tensor} untuk masing-masing \emph{train set}
  dan \emph{test set}.
\end{enumerate}

    \hypertarget{menentukan-train-set-dan-test-set}{%
\subsection{\texorpdfstring{Menentukan \emph{train set} dan \emph{test
set}}{Menentukan train set dan test set}}\label{menentukan-train-set-dan-test-set}}

Sudah direncanakan bahwa untuk \emph{train set} menggunakan periode
selama 8 tahun sedangkan \emph{test set} selama 2 tahun. Sehingga
diperoleh pembagian sebagai berikut:

\begin{itemize}
\tightlist
\item
  \texttt{train\_set}, dari 1 Maret 1998 hingga 31 Desember 2006
  (\textasciitilde{}8 tahun / 3228 hari).
\item
  \texttt{test\_set}, dari 1 Januari 2007 hingga 31 Desember 2008 (2
  tahun / 731 hari).
\end{itemize}

    \begin{Verbatim}[commandchars=\\\{\}]
:: Pemotongan Train set dari    : None     sampai 20061231
:: Pemotongan Test set dari     : 20070101 sampai None
    \end{Verbatim}

    \begin{Verbatim}[commandchars=\\\{\}]
:: DIMENSI TRAIN SET DAN TEST SET
:: DIMENSI [train\_set]: (3228, 14)
:: DIMENSI [test\_set]: (731, 14)
    \end{Verbatim}

    \hypertarget{normalisasi-dataset}{%
\subsection{Normalisasi dataset}\label{normalisasi-dataset}}

Agar pelatihan berlangsung secara efisien, maka seluruh dataset (input
dan output) dinormalisasikan dengan cara mengurangi dengan nilai rerata
dan dibagi oleh standard deviasi (LeCun et al., 2012; Minns and Hall,
1996) sebagaimana disebutkan pada makalah Kratzert et al. (2018).

Normalisasi menggunakan scikit-learn \texttt{StandardScaler}
(\href{https://scikit-learn.org/stable/modules/generated/sklearn.preprocessing.StandardScaler.html}{referensi}).
Parameter objek \texttt{StandardScaler} hanya mengacu pada
\texttt{train\_set}.

    \hypertarget{train-set}{%
\subsubsection{Train set}\label{train-set}}

    \begin{Verbatim}[commandchars=\\\{\}]
:: NORMALISASI DATAFRAME [train\_set] MENJADI [train\_set\_scale]
    \end{Verbatim}

    \begin{Verbatim}[commandchars=\\\{\}]
:: MENAMPILKAN SAMPLE DATASET [train\_set\_scale]
    \end{Verbatim}

            \begin{tcolorbox}[breakable, size=fbox, boxrule=.5pt, pad at break*=1mm, opacityfill=0]
\prompt{Out}{outcolor}{0}{\boxspacing}
\begin{Verbatim}[commandchars=\\\{\}]
                ch\_F  suhu\_rerata      ch\_G      ch\_A      ch\_C
2004-12-20  0.135552     0.652575  0.114218  1.130425 -0.467389
1999-12-17  3.800097    -2.585187  0.490628 -0.472120  1.307047
1998-09-13 -0.410231     0.883843 -0.477285 -0.225575 -0.467389
1999-12-10 -0.176324     0.305672 -0.047101 -0.061211  0.952160
2003-03-31 -0.410231     0.999478 -0.477285 -0.472120 -0.467389
\end{Verbatim}
\end{tcolorbox}
        
    \hypertarget{test-set}{%
\subsubsection{Test set}\label{test-set}}

Normalisasi pada \emph{test set} menggunakan parameter dari \emph{train
set}.

    \begin{Verbatim}[commandchars=\\\{\}]
:: NORMALISASI DATAFRAME [test\_set] MENJADI [test\_set\_scale]
    \end{Verbatim}

    \begin{Verbatim}[commandchars=\\\{\}]
:: MENAMPILKAN SAMPLE DATASET [test\_set\_scale]
    \end{Verbatim}

            \begin{tcolorbox}[breakable, size=fbox, boxrule=.5pt, pad at break*=1mm, opacityfill=0]
\prompt{Out}{outcolor}{0}{\boxspacing}
\begin{Verbatim}[commandchars=\\\{\}]
                ch\_B      ch\_F     debit      ch\_C  suhu\_max
2007-07-12 -0.476333 -0.410231 -0.539789 -0.467389 -0.748948
2008-06-08 -0.476333 -0.410231 -0.529627 -0.467389 -0.028437
2008-05-19 -0.081408  0.837274 -0.516454 -0.467389  0.547973
2008-08-25 -0.476333  0.759305 -0.167470  1.803890 -0.172539
2008-05-06 -0.476333  1.071181 -0.458494  1.023137  0.115666
\end{Verbatim}
\end{tcolorbox}
        
    \hypertarget{input-tensor}{%
\subsection{INPUT TENSOR}\label{input-tensor}}

Setelah melakukan proses normalisasi, maka \texttt{train\_set} harus
ditransformasi ke dalam bentuk tensor 3 dimensi. Dalam pemodelan RNN
dimensi input berupa tensor 3 dimensi sebagai
\texttt{(batch\_size,\ timesteps,\ input\_dim)}
(\href{https://keras.io/layers/recurrent/}{referensi}).

Berdasarkan Kratzert et al. (2018), \textbf{\emph{timesteps} yang
digunakan sebesar \texttt{TIMESTEPS=365} hari}. Nilai tersebut digunakan
untuk dapat menangkap setidaknya siklus tahunan. Pada buku ini juga akan
menggunakan nilai yang sama.

Proses transformasi ini akan menggunakan modul
\texttt{hidrokit.contrib.taruma.hk53}
(\href{https://nbviewer.jupyter.org/gist/taruma/50460ebfaab5a30c41e7f1a1ac0853e2}{panduan}).

    \begin{Verbatim}[commandchars=\\\{\}]
:: MENENTUKAN [TIMESTEPS]
:: [TIMESTEPS] = 365 hari
    \end{Verbatim}

    \begin{Verbatim}[commandchars=\\\{\}]
:: MENENTUKAN INPUT COLUMNS DAN OUTPUT COLUMNS
:: INPUT COLUMNS = ['ch\_A', 'ch\_B', 'ch\_C', 'ch\_D', 'ch\_E', 'ch\_F', 'ch\_G',
'ch\_H', 'suhu\_min', 'suhu\_max', 'suhu\_rerata', 'lembab\_rerata',
'lama\_penyinaran']
:: OUTPUT COLUMNS = ['debit']
    \end{Verbatim}

    \hypertarget{train-set}{%
\subsubsection{Train Set}\label{train-set}}

    \begin{Verbatim}[commandchars=\\\{\}]
:: TRANSFORMASI TRAIN SET
:: DIMENSI [train\_set\_scale] = (3228, 14)
:: TRANSFORMASI [train\_set\_scale] MENJADI [X\_train] DAN [y\_train]
:: DIMENSI [X\_train] = (2863, 365, 13)
:: DIMENSI [y\_train] = (2863,)
    \end{Verbatim}

    \hypertarget{test-set}{%
\subsubsection{Test Set}\label{test-set}}

    \begin{Verbatim}[commandchars=\\\{\}]
:: TRANSFORMASI TEST SET
:: DIMENSI [test\_set\_scale] = (731, 14)
:: TRANSFORMASI [test\_set\_scale] MENJADI [X\_test] DAN [y\_test]
:: DIMENSI [X\_test] = (366, 365, 13)
:: DIMENSI [y\_test] = (366,)
    \end{Verbatim}

    \hypertarget{tahap-4-melatih-model}{%
\section{TAHAP 4: MELATIH MODEL}\label{tahap-4-melatih-model}}

Pada tahap ini, akan mempersiapkan arsitektur RNN/LSTM disertai
melakukan pelatihan (\emph{training}) model.

Demi mempersingkat buku, parameter yang digunakan adalah \textbf{dua
layer dan 20 sel} dengan setiap layer diberi \emph{Dropout} layer dengan
probabilitas 10\% (meniru makalah Kratzert et al. (2018)).

    \begin{Verbatim}[commandchars=\\\{\}]
:: IMPORT TENSORFLOW.KERAS LIBRARY
    \end{Verbatim}

    Dalam bidang hidrologi, kasus curah hujan-limpasan dapat dievaluasi
dengan berbagai metrik, metrik yang biasa digunakan adalah
Nash-Sutcliffe Efficiency
(\href{https://en.wikipedia.org/wiki/Nash–Sutcliffe_model_efficiency_coefficient}{referensi}).
Karena evaluasi metrik ingin dilakukan setiap epoch, maka dari itu
dibuat fungsi khusus agar disertakan saat \texttt{compile} model.

Untuk evaluasi metrik sebenarnya sudah tersedia paket
\href{https://github.com/BYU-Hydroinformatics/HydroErr}{HydroErr} yang
dibuat oleh BYU Hydroinformatics (tersedia juga paket
\href{https://github.com/BYU-Hydroinformatics/Hydrostats}{HydroStats}
untuk menelaah data hidrologi). Karena objek yang diterima pada metrik
\texttt{keras} harus berupa \texttt{tensorflow} harus dibuat fungsi
khusus tersendiri.

    \begin{Verbatim}[commandchars=\\\{\}]
:: MEMBUAT FUNGSI KHUSUS METRIK (NSE, NSE\_MOD, R\_SQUARED)
    \end{Verbatim}

    Pada evaluasi metrik, digunakan empat fungsi yaitu \emph{mean absolute
error} \texttt{mae}, \emph{Nash--Sutcliffe Efficiency} \texttt{nse} ,
\emph{Modified NSE} \texttt{nse\_mod}, \emph{Coefficient of
Determination} \texttt{r\_squared}, dan \emph{mean squared error}
\texttt{mse} sebagai \emph{loss function}.

    \begin{Verbatim}[commandchars=\\\{\}]
:: PEMODELAN RNN
:: SUMMARY [model]:
Model: "sequential"
\_\_\_\_\_\_\_\_\_\_\_\_\_\_\_\_\_\_\_\_\_\_\_\_\_\_\_\_\_\_\_\_\_\_\_\_\_\_\_\_\_\_\_\_\_\_\_\_\_\_\_\_\_\_\_\_\_\_\_\_\_\_\_\_\_
Layer (type)                 Output Shape              Param \#
=================================================================
lstm (LSTM)                  (None, 365, 20)           2720
\_\_\_\_\_\_\_\_\_\_\_\_\_\_\_\_\_\_\_\_\_\_\_\_\_\_\_\_\_\_\_\_\_\_\_\_\_\_\_\_\_\_\_\_\_\_\_\_\_\_\_\_\_\_\_\_\_\_\_\_\_\_\_\_\_
dropout (Dropout)            (None, 365, 20)           0
\_\_\_\_\_\_\_\_\_\_\_\_\_\_\_\_\_\_\_\_\_\_\_\_\_\_\_\_\_\_\_\_\_\_\_\_\_\_\_\_\_\_\_\_\_\_\_\_\_\_\_\_\_\_\_\_\_\_\_\_\_\_\_\_\_
lstm\_1 (LSTM)                (None, 20)                3280
\_\_\_\_\_\_\_\_\_\_\_\_\_\_\_\_\_\_\_\_\_\_\_\_\_\_\_\_\_\_\_\_\_\_\_\_\_\_\_\_\_\_\_\_\_\_\_\_\_\_\_\_\_\_\_\_\_\_\_\_\_\_\_\_\_
dropout\_1 (Dropout)          (None, 20)                0
\_\_\_\_\_\_\_\_\_\_\_\_\_\_\_\_\_\_\_\_\_\_\_\_\_\_\_\_\_\_\_\_\_\_\_\_\_\_\_\_\_\_\_\_\_\_\_\_\_\_\_\_\_\_\_\_\_\_\_\_\_\_\_\_\_
dense (Dense)                (None, 1)                 21
=================================================================
Total params: 6,021
Trainable params: 6,021
Non-trainable params: 0
\_\_\_\_\_\_\_\_\_\_\_\_\_\_\_\_\_\_\_\_\_\_\_\_\_\_\_\_\_\_\_\_\_\_\_\_\_\_\_\_\_\_\_\_\_\_\_\_\_\_\_\_\_\_\_\_\_\_\_\_\_\_\_\_\_
    \end{Verbatim}

    \begin{Verbatim}[commandchars=\\\{\}]
:: TRAINING START: 20191022 08:09
    \end{Verbatim}

    Pada buku ini hanya akan dilakukan sebanyak \texttt{epochs=50} dengan
\texttt{batch\_size=30}.

    \begin{Verbatim}[commandchars=\\\{\}]
:: TRAINING FINISH: 20191022 08:28
    \end{Verbatim}

    \begin{Verbatim}[commandchars=\\\{\}]
:: MODEL SELESAI DILATIH
:: DURASI: 0:18:57.397631
    \end{Verbatim}

    \hypertarget{tahap-5-evaluasi-model}{%
\section{Tahap 5: EVALUASI MODEL}\label{tahap-5-evaluasi-model}}

Tahap ini akan membahas hasil pelatihan model. Langkah yang akan
dilakukan antara lain:

\begin{itemize}
\tightlist
\item
  Mengevaluasi metrik yang telah tercatat dalam \texttt{history}.
\item
  Mengembalikan hasil normalisasi menjadi nilai sebenarnya.
\item
  Mengevaluasi \emph{train set}
\item
  Mengevaluasi \emph{test set}
\end{itemize}

    \hypertarget{metrik}{%
\subsection{Metrik}\label{metrik}}

    Terdapat 5 metrik yang telah tercatat yaitu:

\begin{itemize}
\tightlist
\item
  \emph{mean squared error} yang digunakan sebagai \emph{loss function}
  \texttt{loss}: \(0 \leq MSE \leq \infty\), semakin kecil semakin baik.
\item
  \emph{mean absolute error} \texttt{mae}: \(0 \leq MAE \leq \infty\),
  semakin kecil semakin baik.
\item
  \emph{Nash-Sutcliffe Efficiency} \texttt{nse}:
  \(-\infty \leq NSE < 1\), semakin besar semakin baik.
\item
  \emph{Modified NSE} \texttt{nse\_mod}: \(-\infty \leq NSE\_MOD < 1\),
  semakin besar semakin baik.
\item
  \emph{Coefficient of Determination} \texttt{r\_squared}:
  \(0 \leq R^2 \leq 1\), dengan nilai 1 menandakan data berkorelasi
  sempurna (data prediksi sama persis dengan data sebenarnya).
\end{itemize}

    \begin{Verbatim}[commandchars=\\\{\}]
:: MENYIMPAN HISTORY METRIK DALAM BENTUK DATAFRAME
:: METRIK DISIMPAN DI [df\_metric]
:: MENAMPILKAN [df\_metric]
    \end{Verbatim}

            \begin{tcolorbox}[breakable, size=fbox, boxrule=.5pt, pad at break*=1mm, opacityfill=0]
\prompt{Out}{outcolor}{0}{\boxspacing}
\begin{Verbatim}[commandchars=\\\{\}]
        mse       mae       nse   nse\_mod  r\_squared
0  0.781758  0.484053  0.124985  0.132563   0.235792
1  0.692471  0.448786  0.198317  0.193541   0.324584
2  0.627708  0.432311  0.201291  0.203872   0.370504
3  0.595955  0.428561  0.261062  0.219740   0.407329
4  0.583112  0.417041  0.268653  0.238746   0.410968
\end{Verbatim}
\end{tcolorbox}
        
    \begin{Verbatim}[commandchars=\\\{\}]
:: GRAFIK METRIK
    \end{Verbatim}

    \begin{center}
    \adjustimage{max size={0.9\linewidth}{0.9\paperheight}}{github_taruma_demo_lstm_rr_files/github_taruma_demo_lstm_rr_103_1.png}
    \end{center}
    { \hspace*{\fill} \\}
    
    Dapat dilihat pada grafik, pada setiap epochnya model masih terus
membaik, sehingga dimungkinkan untuk melanjutkan pelatihan lebih dari 50
epoch.

    \hypertarget{object-standardscaler-untuk-kolom-debit}{%
\subsection{\texorpdfstring{Object \texttt{StandardScaler} untuk kolom
debit}{Object StandardScaler untuk kolom debit}}\label{object-standardscaler-untuk-kolom-debit}}

Karena proses normalisasi data dilakukan sebelum pemisahan \emph{train
set} dan \emph{test set}, maka harus dibuat objek
\texttt{StandardScaler} baru yang mengambil atribut objek \texttt{sc}
pada kolom debit (y\_train/y\_test).

    \begin{Verbatim}[commandchars=\\\{\}]
:: OBJEK StandardScaler UNTUK OUTPUT [y\_train] DAN [y\_test]
    \end{Verbatim}

    \hypertarget{evaluasi-train-set}{%
\subsection{\texorpdfstring{Evaluasi \emph{train
set}}{Evaluasi train set}}\label{evaluasi-train-set}}

Meski mengevaluasi train set sudah tersampaikan melalui metrik diatas,
saya ingin melihat bagaimana model berhasil memprediksikan data train
set setiap harinya.

    \begin{Verbatim}[commandchars=\\\{\}]
:: MENYIMPAN EVALUASI TRAIN SET [df\_eval\_train]
:: MENAMPILKAN [df\_eval\_train]
    \end{Verbatim}

            \begin{tcolorbox}[breakable, size=fbox, boxrule=.5pt, pad at break*=1mm, opacityfill=0]
\prompt{Out}{outcolor}{0}{\boxspacing}
\begin{Verbatim}[commandchars=\\\{\}]
              true        pred
1999-03-01  376.00  334.103973
1999-03-02  282.00  301.464172
1999-03-03  188.00  151.581390
1999-03-04   23.15   65.298645
1999-03-05   23.15   54.322853
\end{Verbatim}
\end{tcolorbox}
        
    \begin{Verbatim}[commandchars=\\\{\}]
:: GRAFIK DEBIT ALIRAN MENGGUNAKAN TRAIN SET
    \end{Verbatim}

    \begin{center}
    \adjustimage{max size={0.9\linewidth}{0.9\paperheight}}{github_taruma_demo_lstm_rr_files/github_taruma_demo_lstm_rr_109_1.png}
    \end{center}
    { \hspace*{\fill} \\}
    
    Sejauh ini model mampu memprediksikannya dengan cukup memuaskan. Model
mampu melihat kejadian peningkatan ataupun penurunan debit. Tentunya,
mengevaluasi data \emph{train set} tidak begitu signifikan, dikarenakan
model memang sudah dilatih berdasarkan data \emph{train set}, tidak aneh
jika pemodelannya memuaskan.

    \hypertarget{evaluasi-test-set}{%
\subsection{\texorpdfstring{Evaluasi \emph{test
set}}{Evaluasi test set}}\label{evaluasi-test-set}}

Performa model dapat dievaluasi menggunakan data \emph{test set}, dimana
data tersebut tidak terlihat sama sekali oleh model.

Evaluasi dilakukan dengan mengukur 5 metrik yang dilakukan serupa pada
saat pelatihan (mse, mae, nse, nse\_mod, dan r\_squared). Perhitungan
metrik akan menggunakan paket HydroErr.

    \begin{Verbatim}[commandchars=\\\{\}]
:: MENYIMPAN EVALUASI TEST SET [df\_eval\_test]
:: MENAMPILKAN [df\_eval\_test]
    \end{Verbatim}

            \begin{tcolorbox}[breakable, size=fbox, boxrule=.5pt, pad at break*=1mm, opacityfill=0]
\prompt{Out}{outcolor}{0}{\boxspacing}
\begin{Verbatim}[commandchars=\\\{\}]
              true        pred
2008-01-01   76.03   50.311241
2008-01-02  679.00   56.357487
2008-01-03  632.00  189.934631
2008-01-04  408.00  151.144196
2008-01-05  465.00  209.659058
\end{Verbatim}
\end{tcolorbox}
        
    \begin{Verbatim}[commandchars=\\\{\}]
:: GRAFIK DEBIT ALIRAN MENGGUNAKAN TEST SET
    \end{Verbatim}

    \begin{center}
    \adjustimage{max size={0.9\linewidth}{0.9\paperheight}}{github_taruma_demo_lstm_rr_files/github_taruma_demo_lstm_rr_113_1.png}
    \end{center}
    { \hspace*{\fill} \\}
    
    Hasil prediksi debit aliran menggunakan data \emph{test set} masih
dibilang tidak tahu pasti bagus tidaknya berdasarkan visualisasi.
Terlihat ada beberapa nilai yang meleset dari data observasi, akan
tetapi model mampu memprediksikan kondisi kekeringan yang terjadi pada
bulan juni hingga agustus.

    \begin{Verbatim}[commandchars=\\\{\}]
:: MEMERIKSA PAKET HYDROERR
:: INSTALASI PAKET HYDROERR
  Building wheel for HydroErr (setup.py) {\ldots} done
    \end{Verbatim}

    \begin{Verbatim}[commandchars=\\\{\}]
:: MENGHITUNG METRIK DARI TEST SET
    \end{Verbatim}

    \begin{Verbatim}[commandchars=\\\{\}]
:: MENAMPILKAN METRIK TEST SET [test\_metric]
    \end{Verbatim}

            \begin{tcolorbox}[breakable, size=fbox, boxrule=.5pt, pad at break*=1mm, opacityfill=0]
\prompt{Out}{outcolor}{0}{\boxspacing}
\begin{Verbatim}[commandchars=\\\{\}]
mse          0.725624
mae          0.474683
nse          0.228164
nse\_mod      0.276037
r\_squared    0.273061
dtype: float64
\end{Verbatim}
\end{tcolorbox}
        
    \begin{Verbatim}[commandchars=\\\{\}]
:: MEMBANDINGKAN METRIK TRAIN SET DAN TEST SET
    \end{Verbatim}

            \begin{tcolorbox}[breakable, size=fbox, boxrule=.5pt, pad at break*=1mm, opacityfill=0]
\prompt{Out}{outcolor}{0}{\boxspacing}
\begin{Verbatim}[commandchars=\\\{\}]
              train      test
mse        0.307810  0.725624
mae        0.340651  0.474683
nse        0.510170  0.228164
nse\_mod    0.373428  0.276037
r\_squared  0.586090  0.273061
\end{Verbatim}
\end{tcolorbox}
        
    Berdasarkan hasil metrik diatas, prediksi dari data \emph{test set}
tidak begitu bagus sehingga model bisa dibilang masih kurang dilatih.

    \hypertarget{kesimpulan}{%
\section{KESIMPULAN}\label{kesimpulan}}

    Jika melihat dari hasil perhitungan metrik, prediksi yang dihasilkan
oleh model tergolong tidak bagus, tidak ada salah satu parameter yang
dianggap memuaskan. Akan tetapi, jika dilihat dari grafik antara debit
dan waktu (hidrograf), model mampu memprediksikan fluktuasi debit
(kondisi kekeringan).

Dalam buku ini, hanya menampilkan bagaimana pemodelan menggunakan LSTM
dilakukan saja. Masih banyak langkah yang dapat dilakukan untuk
memperbaiki model sekarang seperti melakukan \emph{parameter tuning},
\emph{model selection}, \emph{feature engineering}. Tentunya, data set
yang digunakan harus diperiksa kembali karena dalam buku ini menggunakan
asumsi sederhana bahwa data set sudah layak pakai.

    \hypertarget{changelog}{%
\section{Changelog}\label{changelog}}

\begin{verbatim}
- 20191022 - 1.0.0 - Initial
\end{verbatim}

\textbf{Copyright © 2019 \href{https://taruma.github.io}{Taruma Sakti
Megariansyah}}

Source code in this notebook is licensed
\href{https://github.com/taruma/vivaldi/blob/master/LICENSE}{MIT
License}. Data in this notebook is licensed
\href{https://creativecommons.org/licenses/by/4.0/}{Creative Common
Attribution 4.0 International}.


    % Add a bibliography block to the postdoc
    
    
    
\end{document}
