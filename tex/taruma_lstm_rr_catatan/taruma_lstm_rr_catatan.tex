\documentclass[11pt]{article}

    \usepackage[breakable]{tcolorbox}
    \usepackage{parskip} % Stop auto-indenting (to mimic markdown behaviour)
    
    \usepackage{iftex}
    \ifPDFTeX
    	\usepackage[T1]{fontenc}
    	\usepackage{mathpazo}
    \else
    	\usepackage{fontspec}
    \fi

    % Basic figure setup, for now with no caption control since it's done
    % automatically by Pandoc (which extracts ![](path) syntax from Markdown).
    \usepackage{graphicx}
    % Maintain compatibility with old templates. Remove in nbconvert 6.0
    \let\Oldincludegraphics\includegraphics
    % Ensure that by default, figures have no caption (until we provide a
    % proper Figure object with a Caption API and a way to capture that
    % in the conversion process - todo).
    \usepackage{caption}
    \DeclareCaptionFormat{nocaption}{}
    \captionsetup{format=nocaption,aboveskip=0pt,belowskip=0pt}

    \usepackage[Export]{adjustbox} % Used to constrain images to a maximum size
    \adjustboxset{max size={0.9\linewidth}{0.9\paperheight}}
    \usepackage{float}
    \floatplacement{figure}{H} % forces figures to be placed at the correct location
    \usepackage{xcolor} % Allow colors to be defined
    \usepackage{enumerate} % Needed for markdown enumerations to work
    \usepackage{geometry} % Used to adjust the document margins
    \usepackage{amsmath} % Equations
    \usepackage{amssymb} % Equations
    \usepackage{textcomp} % defines textquotesingle
    % Hack from http://tex.stackexchange.com/a/47451/13684:
    \AtBeginDocument{%
        \def\PYZsq{\textquotesingle}% Upright quotes in Pygmentized code
    }
    \usepackage{upquote} % Upright quotes for verbatim code
    \usepackage{eurosym} % defines \euro
    \usepackage[mathletters]{ucs} % Extended unicode (utf-8) support
    \usepackage{fancyvrb} % verbatim replacement that allows latex
    \usepackage{grffile} % extends the file name processing of package graphics 
                         % to support a larger range
    \makeatletter % fix for grffile with XeLaTeX
    \def\Gread@@xetex#1{%
      \IfFileExists{"\Gin@base".bb}%
      {\Gread@eps{\Gin@base.bb}}%
      {\Gread@@xetex@aux#1}%
    }
    \makeatother

    % The hyperref package gives us a pdf with properly built
    % internal navigation ('pdf bookmarks' for the table of contents,
    % internal cross-reference links, web links for URLs, etc.)
    \usepackage{hyperref}
    % The default LaTeX title has an obnoxious amount of whitespace. By default,
    % titling removes some of it. It also provides customization options.
    \usepackage{titling}
    \usepackage{longtable} % longtable support required by pandoc >1.10
    \usepackage{booktabs}  % table support for pandoc > 1.12.2
    \usepackage[inline]{enumitem} % IRkernel/repr support (it uses the enumerate* environment)
    \usepackage[normalem]{ulem} % ulem is needed to support strikethroughs (\sout)
                                % normalem makes italics be italics, not underlines
    \usepackage{mathrsfs}
    

    
    % Colors for the hyperref package
    \definecolor{urlcolor}{rgb}{0,.145,.698}
    \definecolor{linkcolor}{rgb}{.71,0.21,0.01}
    \definecolor{citecolor}{rgb}{.12,.54,.11}

    % ANSI colors
    \definecolor{ansi-black}{HTML}{3E424D}
    \definecolor{ansi-black-intense}{HTML}{282C36}
    \definecolor{ansi-red}{HTML}{E75C58}
    \definecolor{ansi-red-intense}{HTML}{B22B31}
    \definecolor{ansi-green}{HTML}{00A250}
    \definecolor{ansi-green-intense}{HTML}{007427}
    \definecolor{ansi-yellow}{HTML}{DDB62B}
    \definecolor{ansi-yellow-intense}{HTML}{B27D12}
    \definecolor{ansi-blue}{HTML}{208FFB}
    \definecolor{ansi-blue-intense}{HTML}{0065CA}
    \definecolor{ansi-magenta}{HTML}{D160C4}
    \definecolor{ansi-magenta-intense}{HTML}{A03196}
    \definecolor{ansi-cyan}{HTML}{60C6C8}
    \definecolor{ansi-cyan-intense}{HTML}{258F8F}
    \definecolor{ansi-white}{HTML}{C5C1B4}
    \definecolor{ansi-white-intense}{HTML}{A1A6B2}
    \definecolor{ansi-default-inverse-fg}{HTML}{FFFFFF}
    \definecolor{ansi-default-inverse-bg}{HTML}{000000}

    % commands and environments needed by pandoc snippets
    % extracted from the output of `pandoc -s`
    \providecommand{\tightlist}{%
      \setlength{\itemsep}{0pt}\setlength{\parskip}{0pt}}
    \DefineVerbatimEnvironment{Highlighting}{Verbatim}{commandchars=\\\{\}}
    % Add ',fontsize=\small' for more characters per line
    \newenvironment{Shaded}{}{}
    \newcommand{\KeywordTok}[1]{\textcolor[rgb]{0.00,0.44,0.13}{\textbf{{#1}}}}
    \newcommand{\DataTypeTok}[1]{\textcolor[rgb]{0.56,0.13,0.00}{{#1}}}
    \newcommand{\DecValTok}[1]{\textcolor[rgb]{0.25,0.63,0.44}{{#1}}}
    \newcommand{\BaseNTok}[1]{\textcolor[rgb]{0.25,0.63,0.44}{{#1}}}
    \newcommand{\FloatTok}[1]{\textcolor[rgb]{0.25,0.63,0.44}{{#1}}}
    \newcommand{\CharTok}[1]{\textcolor[rgb]{0.25,0.44,0.63}{{#1}}}
    \newcommand{\StringTok}[1]{\textcolor[rgb]{0.25,0.44,0.63}{{#1}}}
    \newcommand{\CommentTok}[1]{\textcolor[rgb]{0.38,0.63,0.69}{\textit{{#1}}}}
    \newcommand{\OtherTok}[1]{\textcolor[rgb]{0.00,0.44,0.13}{{#1}}}
    \newcommand{\AlertTok}[1]{\textcolor[rgb]{1.00,0.00,0.00}{\textbf{{#1}}}}
    \newcommand{\FunctionTok}[1]{\textcolor[rgb]{0.02,0.16,0.49}{{#1}}}
    \newcommand{\RegionMarkerTok}[1]{{#1}}
    \newcommand{\ErrorTok}[1]{\textcolor[rgb]{1.00,0.00,0.00}{\textbf{{#1}}}}
    \newcommand{\NormalTok}[1]{{#1}}
    
    % Additional commands for more recent versions of Pandoc
    \newcommand{\ConstantTok}[1]{\textcolor[rgb]{0.53,0.00,0.00}{{#1}}}
    \newcommand{\SpecialCharTok}[1]{\textcolor[rgb]{0.25,0.44,0.63}{{#1}}}
    \newcommand{\VerbatimStringTok}[1]{\textcolor[rgb]{0.25,0.44,0.63}{{#1}}}
    \newcommand{\SpecialStringTok}[1]{\textcolor[rgb]{0.73,0.40,0.53}{{#1}}}
    \newcommand{\ImportTok}[1]{{#1}}
    \newcommand{\DocumentationTok}[1]{\textcolor[rgb]{0.73,0.13,0.13}{\textit{{#1}}}}
    \newcommand{\AnnotationTok}[1]{\textcolor[rgb]{0.38,0.63,0.69}{\textbf{\textit{{#1}}}}}
    \newcommand{\CommentVarTok}[1]{\textcolor[rgb]{0.38,0.63,0.69}{\textbf{\textit{{#1}}}}}
    \newcommand{\VariableTok}[1]{\textcolor[rgb]{0.10,0.09,0.49}{{#1}}}
    \newcommand{\ControlFlowTok}[1]{\textcolor[rgb]{0.00,0.44,0.13}{\textbf{{#1}}}}
    \newcommand{\OperatorTok}[1]{\textcolor[rgb]{0.40,0.40,0.40}{{#1}}}
    \newcommand{\BuiltInTok}[1]{{#1}}
    \newcommand{\ExtensionTok}[1]{{#1}}
    \newcommand{\PreprocessorTok}[1]{\textcolor[rgb]{0.74,0.48,0.00}{{#1}}}
    \newcommand{\AttributeTok}[1]{\textcolor[rgb]{0.49,0.56,0.16}{{#1}}}
    \newcommand{\InformationTok}[1]{\textcolor[rgb]{0.38,0.63,0.69}{\textbf{\textit{{#1}}}}}
    \newcommand{\WarningTok}[1]{\textcolor[rgb]{0.38,0.63,0.69}{\textbf{\textit{{#1}}}}}
    
    
    % Define a nice break command that doesn't care if a line doesn't already
    % exist.
    \def\br{\hspace*{\fill} \\* }
    % Math Jax compatibility definitions
    \def\gt{>}
    \def\lt{<}
    \let\Oldtex\TeX
    \let\Oldlatex\LaTeX
    \renewcommand{\TeX}{\textrm{\Oldtex}}
    \renewcommand{\LaTeX}{\textrm{\Oldlatex}}
    % Document parameters
    % Document title
    \title{github\_taruma\_demo\_lstm\_rr\_catatan}
    
    
    
    
    
% Pygments definitions
\makeatletter
\def\PY@reset{\let\PY@it=\relax \let\PY@bf=\relax%
    \let\PY@ul=\relax \let\PY@tc=\relax%
    \let\PY@bc=\relax \let\PY@ff=\relax}
\def\PY@tok#1{\csname PY@tok@#1\endcsname}
\def\PY@toks#1+{\ifx\relax#1\empty\else%
    \PY@tok{#1}\expandafter\PY@toks\fi}
\def\PY@do#1{\PY@bc{\PY@tc{\PY@ul{%
    \PY@it{\PY@bf{\PY@ff{#1}}}}}}}
\def\PY#1#2{\PY@reset\PY@toks#1+\relax+\PY@do{#2}}

\expandafter\def\csname PY@tok@w\endcsname{\def\PY@tc##1{\textcolor[rgb]{0.73,0.73,0.73}{##1}}}
\expandafter\def\csname PY@tok@c\endcsname{\let\PY@it=\textit\def\PY@tc##1{\textcolor[rgb]{0.25,0.50,0.50}{##1}}}
\expandafter\def\csname PY@tok@cp\endcsname{\def\PY@tc##1{\textcolor[rgb]{0.74,0.48,0.00}{##1}}}
\expandafter\def\csname PY@tok@k\endcsname{\let\PY@bf=\textbf\def\PY@tc##1{\textcolor[rgb]{0.00,0.50,0.00}{##1}}}
\expandafter\def\csname PY@tok@kp\endcsname{\def\PY@tc##1{\textcolor[rgb]{0.00,0.50,0.00}{##1}}}
\expandafter\def\csname PY@tok@kt\endcsname{\def\PY@tc##1{\textcolor[rgb]{0.69,0.00,0.25}{##1}}}
\expandafter\def\csname PY@tok@o\endcsname{\def\PY@tc##1{\textcolor[rgb]{0.40,0.40,0.40}{##1}}}
\expandafter\def\csname PY@tok@ow\endcsname{\let\PY@bf=\textbf\def\PY@tc##1{\textcolor[rgb]{0.67,0.13,1.00}{##1}}}
\expandafter\def\csname PY@tok@nb\endcsname{\def\PY@tc##1{\textcolor[rgb]{0.00,0.50,0.00}{##1}}}
\expandafter\def\csname PY@tok@nf\endcsname{\def\PY@tc##1{\textcolor[rgb]{0.00,0.00,1.00}{##1}}}
\expandafter\def\csname PY@tok@nc\endcsname{\let\PY@bf=\textbf\def\PY@tc##1{\textcolor[rgb]{0.00,0.00,1.00}{##1}}}
\expandafter\def\csname PY@tok@nn\endcsname{\let\PY@bf=\textbf\def\PY@tc##1{\textcolor[rgb]{0.00,0.00,1.00}{##1}}}
\expandafter\def\csname PY@tok@ne\endcsname{\let\PY@bf=\textbf\def\PY@tc##1{\textcolor[rgb]{0.82,0.25,0.23}{##1}}}
\expandafter\def\csname PY@tok@nv\endcsname{\def\PY@tc##1{\textcolor[rgb]{0.10,0.09,0.49}{##1}}}
\expandafter\def\csname PY@tok@no\endcsname{\def\PY@tc##1{\textcolor[rgb]{0.53,0.00,0.00}{##1}}}
\expandafter\def\csname PY@tok@nl\endcsname{\def\PY@tc##1{\textcolor[rgb]{0.63,0.63,0.00}{##1}}}
\expandafter\def\csname PY@tok@ni\endcsname{\let\PY@bf=\textbf\def\PY@tc##1{\textcolor[rgb]{0.60,0.60,0.60}{##1}}}
\expandafter\def\csname PY@tok@na\endcsname{\def\PY@tc##1{\textcolor[rgb]{0.49,0.56,0.16}{##1}}}
\expandafter\def\csname PY@tok@nt\endcsname{\let\PY@bf=\textbf\def\PY@tc##1{\textcolor[rgb]{0.00,0.50,0.00}{##1}}}
\expandafter\def\csname PY@tok@nd\endcsname{\def\PY@tc##1{\textcolor[rgb]{0.67,0.13,1.00}{##1}}}
\expandafter\def\csname PY@tok@s\endcsname{\def\PY@tc##1{\textcolor[rgb]{0.73,0.13,0.13}{##1}}}
\expandafter\def\csname PY@tok@sd\endcsname{\let\PY@it=\textit\def\PY@tc##1{\textcolor[rgb]{0.73,0.13,0.13}{##1}}}
\expandafter\def\csname PY@tok@si\endcsname{\let\PY@bf=\textbf\def\PY@tc##1{\textcolor[rgb]{0.73,0.40,0.53}{##1}}}
\expandafter\def\csname PY@tok@se\endcsname{\let\PY@bf=\textbf\def\PY@tc##1{\textcolor[rgb]{0.73,0.40,0.13}{##1}}}
\expandafter\def\csname PY@tok@sr\endcsname{\def\PY@tc##1{\textcolor[rgb]{0.73,0.40,0.53}{##1}}}
\expandafter\def\csname PY@tok@ss\endcsname{\def\PY@tc##1{\textcolor[rgb]{0.10,0.09,0.49}{##1}}}
\expandafter\def\csname PY@tok@sx\endcsname{\def\PY@tc##1{\textcolor[rgb]{0.00,0.50,0.00}{##1}}}
\expandafter\def\csname PY@tok@m\endcsname{\def\PY@tc##1{\textcolor[rgb]{0.40,0.40,0.40}{##1}}}
\expandafter\def\csname PY@tok@gh\endcsname{\let\PY@bf=\textbf\def\PY@tc##1{\textcolor[rgb]{0.00,0.00,0.50}{##1}}}
\expandafter\def\csname PY@tok@gu\endcsname{\let\PY@bf=\textbf\def\PY@tc##1{\textcolor[rgb]{0.50,0.00,0.50}{##1}}}
\expandafter\def\csname PY@tok@gd\endcsname{\def\PY@tc##1{\textcolor[rgb]{0.63,0.00,0.00}{##1}}}
\expandafter\def\csname PY@tok@gi\endcsname{\def\PY@tc##1{\textcolor[rgb]{0.00,0.63,0.00}{##1}}}
\expandafter\def\csname PY@tok@gr\endcsname{\def\PY@tc##1{\textcolor[rgb]{1.00,0.00,0.00}{##1}}}
\expandafter\def\csname PY@tok@ge\endcsname{\let\PY@it=\textit}
\expandafter\def\csname PY@tok@gs\endcsname{\let\PY@bf=\textbf}
\expandafter\def\csname PY@tok@gp\endcsname{\let\PY@bf=\textbf\def\PY@tc##1{\textcolor[rgb]{0.00,0.00,0.50}{##1}}}
\expandafter\def\csname PY@tok@go\endcsname{\def\PY@tc##1{\textcolor[rgb]{0.53,0.53,0.53}{##1}}}
\expandafter\def\csname PY@tok@gt\endcsname{\def\PY@tc##1{\textcolor[rgb]{0.00,0.27,0.87}{##1}}}
\expandafter\def\csname PY@tok@err\endcsname{\def\PY@bc##1{\setlength{\fboxsep}{0pt}\fcolorbox[rgb]{1.00,0.00,0.00}{1,1,1}{\strut ##1}}}
\expandafter\def\csname PY@tok@kc\endcsname{\let\PY@bf=\textbf\def\PY@tc##1{\textcolor[rgb]{0.00,0.50,0.00}{##1}}}
\expandafter\def\csname PY@tok@kd\endcsname{\let\PY@bf=\textbf\def\PY@tc##1{\textcolor[rgb]{0.00,0.50,0.00}{##1}}}
\expandafter\def\csname PY@tok@kn\endcsname{\let\PY@bf=\textbf\def\PY@tc##1{\textcolor[rgb]{0.00,0.50,0.00}{##1}}}
\expandafter\def\csname PY@tok@kr\endcsname{\let\PY@bf=\textbf\def\PY@tc##1{\textcolor[rgb]{0.00,0.50,0.00}{##1}}}
\expandafter\def\csname PY@tok@bp\endcsname{\def\PY@tc##1{\textcolor[rgb]{0.00,0.50,0.00}{##1}}}
\expandafter\def\csname PY@tok@fm\endcsname{\def\PY@tc##1{\textcolor[rgb]{0.00,0.00,1.00}{##1}}}
\expandafter\def\csname PY@tok@vc\endcsname{\def\PY@tc##1{\textcolor[rgb]{0.10,0.09,0.49}{##1}}}
\expandafter\def\csname PY@tok@vg\endcsname{\def\PY@tc##1{\textcolor[rgb]{0.10,0.09,0.49}{##1}}}
\expandafter\def\csname PY@tok@vi\endcsname{\def\PY@tc##1{\textcolor[rgb]{0.10,0.09,0.49}{##1}}}
\expandafter\def\csname PY@tok@vm\endcsname{\def\PY@tc##1{\textcolor[rgb]{0.10,0.09,0.49}{##1}}}
\expandafter\def\csname PY@tok@sa\endcsname{\def\PY@tc##1{\textcolor[rgb]{0.73,0.13,0.13}{##1}}}
\expandafter\def\csname PY@tok@sb\endcsname{\def\PY@tc##1{\textcolor[rgb]{0.73,0.13,0.13}{##1}}}
\expandafter\def\csname PY@tok@sc\endcsname{\def\PY@tc##1{\textcolor[rgb]{0.73,0.13,0.13}{##1}}}
\expandafter\def\csname PY@tok@dl\endcsname{\def\PY@tc##1{\textcolor[rgb]{0.73,0.13,0.13}{##1}}}
\expandafter\def\csname PY@tok@s2\endcsname{\def\PY@tc##1{\textcolor[rgb]{0.73,0.13,0.13}{##1}}}
\expandafter\def\csname PY@tok@sh\endcsname{\def\PY@tc##1{\textcolor[rgb]{0.73,0.13,0.13}{##1}}}
\expandafter\def\csname PY@tok@s1\endcsname{\def\PY@tc##1{\textcolor[rgb]{0.73,0.13,0.13}{##1}}}
\expandafter\def\csname PY@tok@mb\endcsname{\def\PY@tc##1{\textcolor[rgb]{0.40,0.40,0.40}{##1}}}
\expandafter\def\csname PY@tok@mf\endcsname{\def\PY@tc##1{\textcolor[rgb]{0.40,0.40,0.40}{##1}}}
\expandafter\def\csname PY@tok@mh\endcsname{\def\PY@tc##1{\textcolor[rgb]{0.40,0.40,0.40}{##1}}}
\expandafter\def\csname PY@tok@mi\endcsname{\def\PY@tc##1{\textcolor[rgb]{0.40,0.40,0.40}{##1}}}
\expandafter\def\csname PY@tok@il\endcsname{\def\PY@tc##1{\textcolor[rgb]{0.40,0.40,0.40}{##1}}}
\expandafter\def\csname PY@tok@mo\endcsname{\def\PY@tc##1{\textcolor[rgb]{0.40,0.40,0.40}{##1}}}
\expandafter\def\csname PY@tok@ch\endcsname{\let\PY@it=\textit\def\PY@tc##1{\textcolor[rgb]{0.25,0.50,0.50}{##1}}}
\expandafter\def\csname PY@tok@cm\endcsname{\let\PY@it=\textit\def\PY@tc##1{\textcolor[rgb]{0.25,0.50,0.50}{##1}}}
\expandafter\def\csname PY@tok@cpf\endcsname{\let\PY@it=\textit\def\PY@tc##1{\textcolor[rgb]{0.25,0.50,0.50}{##1}}}
\expandafter\def\csname PY@tok@c1\endcsname{\let\PY@it=\textit\def\PY@tc##1{\textcolor[rgb]{0.25,0.50,0.50}{##1}}}
\expandafter\def\csname PY@tok@cs\endcsname{\let\PY@it=\textit\def\PY@tc##1{\textcolor[rgb]{0.25,0.50,0.50}{##1}}}

\def\PYZbs{\char`\\}
\def\PYZus{\char`\_}
\def\PYZob{\char`\{}
\def\PYZcb{\char`\}}
\def\PYZca{\char`\^}
\def\PYZam{\char`\&}
\def\PYZlt{\char`\<}
\def\PYZgt{\char`\>}
\def\PYZsh{\char`\#}
\def\PYZpc{\char`\%}
\def\PYZdl{\char`\$}
\def\PYZhy{\char`\-}
\def\PYZsq{\char`\'}
\def\PYZdq{\char`\"}
\def\PYZti{\char`\~}
% for compatibility with earlier versions
\def\PYZat{@}
\def\PYZlb{[}
\def\PYZrb{]}
\makeatother


    % For linebreaks inside Verbatim environment from package fancyvrb. 
    \makeatletter
        \newbox\Wrappedcontinuationbox 
        \newbox\Wrappedvisiblespacebox 
        \newcommand*\Wrappedvisiblespace {\textcolor{red}{\textvisiblespace}} 
        \newcommand*\Wrappedcontinuationsymbol {\textcolor{red}{\llap{\tiny$\m@th\hookrightarrow$}}} 
        \newcommand*\Wrappedcontinuationindent {3ex } 
        \newcommand*\Wrappedafterbreak {\kern\Wrappedcontinuationindent\copy\Wrappedcontinuationbox} 
        % Take advantage of the already applied Pygments mark-up to insert 
        % potential linebreaks for TeX processing. 
        %        {, <, #, %, $, ' and ": go to next line. 
        %        _, }, ^, &, >, - and ~: stay at end of broken line. 
        % Use of \textquotesingle for straight quote. 
        \newcommand*\Wrappedbreaksatspecials {% 
            \def\PYGZus{\discretionary{\char`\_}{\Wrappedafterbreak}{\char`\_}}% 
            \def\PYGZob{\discretionary{}{\Wrappedafterbreak\char`\{}{\char`\{}}% 
            \def\PYGZcb{\discretionary{\char`\}}{\Wrappedafterbreak}{\char`\}}}% 
            \def\PYGZca{\discretionary{\char`\^}{\Wrappedafterbreak}{\char`\^}}% 
            \def\PYGZam{\discretionary{\char`\&}{\Wrappedafterbreak}{\char`\&}}% 
            \def\PYGZlt{\discretionary{}{\Wrappedafterbreak\char`\<}{\char`\<}}% 
            \def\PYGZgt{\discretionary{\char`\>}{\Wrappedafterbreak}{\char`\>}}% 
            \def\PYGZsh{\discretionary{}{\Wrappedafterbreak\char`\#}{\char`\#}}% 
            \def\PYGZpc{\discretionary{}{\Wrappedafterbreak\char`\%}{\char`\%}}% 
            \def\PYGZdl{\discretionary{}{\Wrappedafterbreak\char`\$}{\char`\$}}% 
            \def\PYGZhy{\discretionary{\char`\-}{\Wrappedafterbreak}{\char`\-}}% 
            \def\PYGZsq{\discretionary{}{\Wrappedafterbreak\textquotesingle}{\textquotesingle}}% 
            \def\PYGZdq{\discretionary{}{\Wrappedafterbreak\char`\"}{\char`\"}}% 
            \def\PYGZti{\discretionary{\char`\~}{\Wrappedafterbreak}{\char`\~}}% 
        } 
        % Some characters . , ; ? ! / are not pygmentized. 
        % This macro makes them "active" and they will insert potential linebreaks 
        \newcommand*\Wrappedbreaksatpunct {% 
            \lccode`\~`\.\lowercase{\def~}{\discretionary{\hbox{\char`\.}}{\Wrappedafterbreak}{\hbox{\char`\.}}}% 
            \lccode`\~`\,\lowercase{\def~}{\discretionary{\hbox{\char`\,}}{\Wrappedafterbreak}{\hbox{\char`\,}}}% 
            \lccode`\~`\;\lowercase{\def~}{\discretionary{\hbox{\char`\;}}{\Wrappedafterbreak}{\hbox{\char`\;}}}% 
            \lccode`\~`\:\lowercase{\def~}{\discretionary{\hbox{\char`\:}}{\Wrappedafterbreak}{\hbox{\char`\:}}}% 
            \lccode`\~`\?\lowercase{\def~}{\discretionary{\hbox{\char`\?}}{\Wrappedafterbreak}{\hbox{\char`\?}}}% 
            \lccode`\~`\!\lowercase{\def~}{\discretionary{\hbox{\char`\!}}{\Wrappedafterbreak}{\hbox{\char`\!}}}% 
            \lccode`\~`\/\lowercase{\def~}{\discretionary{\hbox{\char`\/}}{\Wrappedafterbreak}{\hbox{\char`\/}}}% 
            \catcode`\.\active
            \catcode`\,\active 
            \catcode`\;\active
            \catcode`\:\active
            \catcode`\?\active
            \catcode`\!\active
            \catcode`\/\active 
            \lccode`\~`\~ 	
        }
    \makeatother

    \let\OriginalVerbatim=\Verbatim
    \makeatletter
    \renewcommand{\Verbatim}[1][1]{%
        %\parskip\z@skip
        \sbox\Wrappedcontinuationbox {\Wrappedcontinuationsymbol}%
        \sbox\Wrappedvisiblespacebox {\FV@SetupFont\Wrappedvisiblespace}%
        \def\FancyVerbFormatLine ##1{\hsize\linewidth
            \vtop{\raggedright\hyphenpenalty\z@\exhyphenpenalty\z@
                \doublehyphendemerits\z@\finalhyphendemerits\z@
                \strut ##1\strut}%
        }%
        % If the linebreak is at a space, the latter will be displayed as visible
        % space at end of first line, and a continuation symbol starts next line.
        % Stretch/shrink are however usually zero for typewriter font.
        \def\FV@Space {%
            \nobreak\hskip\z@ plus\fontdimen3\font minus\fontdimen4\font
            \discretionary{\copy\Wrappedvisiblespacebox}{\Wrappedafterbreak}
            {\kern\fontdimen2\font}%
        }%
        
        % Allow breaks at special characters using \PYG... macros.
        \Wrappedbreaksatspecials
        % Breaks at punctuation characters . , ; ? ! and / need catcode=\active 	
        \OriginalVerbatim[#1,codes*=\Wrappedbreaksatpunct]%
    }
    \makeatother

    % Exact colors from NB
    \definecolor{incolor}{HTML}{303F9F}
    \definecolor{outcolor}{HTML}{D84315}
    \definecolor{cellborder}{HTML}{CFCFCF}
    \definecolor{cellbackground}{HTML}{F7F7F7}
    
    % prompt
    \makeatletter
    \newcommand{\boxspacing}{\kern\kvtcb@left@rule\kern\kvtcb@boxsep}
    \makeatother
    \newcommand{\prompt}[4]{
        \ttfamily\llap{{\color{#2}[#3]:\hspace{3pt}#4}}\vspace{-\baselineskip}
    }
    

    
    % Prevent overflowing lines due to hard-to-break entities
    \sloppy 
    % Setup hyperref package
    \hypersetup{
      breaklinks=true,  % so long urls are correctly broken across lines
      colorlinks=true,
      urlcolor=urlcolor,
      linkcolor=linkcolor,
      citecolor=citecolor,
      }
    % Slightly bigger margins than the latex defaults
    
    \geometry{verbose,tmargin=1in,bmargin=1in,lmargin=1in,rmargin=1in}
    
    

\begin{document}
  
%    \maketitle
%	ref: https://stackoverflow.com/questions/3141702/vertically-centering-a-title-page
	\begin{titlepage}
		\vspace*{\fill}
		\begin{center}
 		\normalsize Catatan Laporan\\
		\huge Prediksi Debit Aliran menggunakan \emph{Long Short-Term Memory} (LSTM)\\ 
		\normalsize Versi 1.0.0 \\[0.2cm]
      	\small Berdasarkan \emph{Jupyter Notebook}: \texttt{github\_taruma\_demo\_lstm\_rr\_catatan.ipynb} \\[0.5cm]
		\normalsize oleh Taruma Sakti Megariansyah\\[0.5cm]
      	\normalsize 22 Oktober 2019\\[1cm]
    	\adjustimage{max size={0.9\linewidth}{1cm}}{vivaldi_logo.png}\\
      	\normalsize github.com/taruma/vivaldi
		\end{center}
    	\vspace*{\fill}
	\end{titlepage}
    
    

    
    Dokumen ini merupakan catatan untuk laporan ``Prediksi Debit Aliran
Menggunakan Metode \emph{Long Short-Term Memory} (LSTM)'' atau berkas
\texttt{github-taruma\_demo\_lstm\_rr.ipynb}.

    \hypertarget{info-dataset}{%
\section{Info Dataset}\label{info-dataset}}

Dataset beserta informasinya diperoleh dari skripsi saya sendiri
berjudul ``Kajian Penerapan Model NRECA di Bendung Pamarayan'' pada
tahun 2015. Data curah hujan dan debit diperoleh dari skripsi. Untuk
data klimatologi, diunduh melalui Data Online BMKG yang diakses pada 2
Oktober 2019, dikarenakan data dari stasiun terdekat tidak lengkap. Saya
akan mengusahakan menyertakan segala informasi mengenai dataset yang
perlu diketahui di dalam catatan ini.

    \hypertarget{dataset}{%
\subsection{Dataset}\label{dataset}}

Dataset merupakan data hidrologi dan klimatologi \textbf{harian} dari
tanggal \textbf{1 Maret 1998} sampai \textbf{31 Desember 2008} (3959
hari). Dataset terpisah menjadi 3 kategori yaitu: data curah hujan, data
klimatologi, dan data debit.

\begin{itemize}
\tightlist
\item
  Data curah hujan diperoleh dari 8 stasiun yaitu:
  \texttt{bojong\_manik}, \texttt{gunung\_tunggal}, \texttt{pasir\_ona},
  \texttt{sampang\_peundeuy}, \texttt{cimarga}, \texttt{bd\_pamarayan},
  \texttt{ciminyak\_cilaki}, \texttt{gardu\_tanjak}.
\item
  Data debit diperoleh dari 1 stasiun yaitu: \texttt{bd\_pamarayan}.
\item
  Data klimatologi diperoleh dari 1 stasiun yaitu:
  \texttt{geofisika\_serang}.
\end{itemize}

    \hypertarget{sumber-dataset}{%
\subsection{Sumber Dataset}\label{sumber-dataset}}

Berikut sumber dataset yang diperoleh (Megariansyah, 2015):

\begin{itemize}
\tightlist
\item
  Data Curah Hujan, 8 Stasiun: BBWS Cidanau-Ciujung-Cidurian
\item
  Data Debit, 1 Stasiun: BBWS Cidanau-Ciujung-Cidurian
\item
  Data Klimatologi, 1 Stasiun: Data Online BMKG
\end{itemize}

    \hypertarget{ringkasan-dataset}{%
\subsection{Ringkasan Dataset}\label{ringkasan-dataset}}

\begin{itemize}
\tightlist
\item
  Data curah hujan merupakan data berkolom tunggal yang menunjukkan
  besarnya curah hujan dalam satuan \(mm\) untuk masing-masing stasiun.
\item
  Data debit merupakan data berkolom tunggal yang menunjukkan besarnya
  debit dalam satuan \(m^3/s\).
\item
  Data klimatologi merupakan data dengan 10 kolom berupa:

  \begin{itemize}
  \tightlist
  \item
    Arah angin saat kecepatan maksimum (ddd\_x) dalam satuan derajat
  \item
    Arah angin terbanyak (ddd\_car) dalam satuan derajat
  \item
    Curah hujan (RR) dalam satuan \(mm\)
  \item
    Kecepatan angin maksimum (ff\_x) dalam satuan \(m/s\)
  \item
    Kecepatan angin rata-rata (ff\_avg) dalam satuan \(m/s\)
  \item
    Kelembapan rata-rata (RH\_avg) dalam satuan \%
  \item
    Lamanya penyinaran matahari (ss) dalam satuan jam
  \item
    Temperatur maksimum (Tx) dalam derajat Celcius
  \item
    Temperatur minimum (Tn) dalam derajat Celcius
  \item
    Temperatur rata-rata (Tavg) dalam derajat Celcius
  \end{itemize}
\item
  Data debit merupakan variabel dependen, sedangkan data lainnya
  merupakan variabel independen.
\item
  Pada data klimatologi, isian yang bernilai \texttt{8888} berarti data
  tidak diukur dan isian yang bernilai \texttt{9999} berarti tidak ada
  data (tidak dilakukan pengukuran). Nilai tersebut akan dianggap nilai
  yang hilang ``\texttt{NaN}''.
\end{itemize}

    \hypertarget{strategi-penyelesaian}{%
\section{Strategi Penyelesaian}\label{strategi-penyelesaian}}

Terdapat 5 tahap yang saya ikuti dalam menjawab objektif buku ini.

\begin{enumerate}
\def\labelenumi{\arabic{enumi}.}
\item
  Tahap 0: Pengaturan Awal dan Inisiasi

  Pada tahap ini dilakukan pengaturan awal dan inisiasi untuk
  mempersiapkan buku. Buku dapat dijalankan secara lokal ataupun
  \emph{cloud} menggunakan Google Colab. Di tahap ini, dapat dilakukan
  pengaturan manual seperti penamaan buku (jika ingin dilakukan
  penyimpanan), menentukan lokasi dataset dan dropbox, dll.
\item
  Tahap 1: Akusisi Dataset

  Dataset yang diterima bisa dalam berbagai bentuk seperti dalam bentuk
  Excel, PDF, bahkan fisik berupa lembaran/laporan. Pada tahap ini
  dilakukan pengubahan dataset tersebut biar bisa diolah secara digital.
  Untungnya, pada buku ini, dataset yang diperoleh berupa digital dengan
  format .xls sehingga memudahkan dalam mempersiapkan pengolahan data
  lebih lanjut.

  Untuk membantu tahap ini juga dibuat modul khusus yang telah tersedia
  di hidrokit yang dapat diakses melalui
  \texttt{hidrokit.contrib.taruma} dengan nama modul \texttt{hk43} untuk
  data hujan/debit dan \texttt{hk73} untuk data klimatologi/bmkg.
\item
  Tahap 2: Prapemrosesan Data

  Tahap ini memastikan kelengkapan data dan validitas data.
  Prapemrosesan dapat berupa mencari nilai invalid dan mengoreksinya,
  memeriksa data yang hilang dan dikoreksi dengan berbagai metode (pada
  buku ini menggunakan interpolasi linear). Karena pemodelan bergantung
  dengan data yang digunakan, tahap ini memiliki peran penting dalam
  keberhasilan pemodelan.
\item
  Tahap 3: Input Pemodelan

  Data yang telah melewati tahap prapemrosesan akan dipersiapkan untuk
  digunakan dalam pemodelan. Persiapan ini berupa memisahkan dataset
  menjadi dua bagian yaitu \emph{train set} dan \emph{test set},
  normalisasi, dan transformasi dataset.

  Pada pemodelan \emph{Recurrent Neural Networks}, input yang diterima
  berbentuk tensor 3D. Pada manual Keras, disebutkan bahwa dimensi
  tensor 3D berupa (\emph{batch\_size}, \emph{timesteps},
  \emph{input\_dim}).

  Dalam buku ini, digunakan \texttt{TIMESTEPS=365} hari serupa pada
  makalah Kratzert et. al. (2018). Nilai \emph{timesteps} tidak harus
  bernilai \texttt{365} di buku ini, nilai \emph{timesteps} dapat di isi
  dengan nilai sembarang sampai memperoleh nilai optimal untuk model.

  Untuk membantu tahap ini, dibuat modul khusus yang dapat diakses
  melalui \texttt{hidrokit.contrib.taruma.hk53}.
\item
  Tahap 4: Melatih Model

  Pada tahap ini harus ditentukan arsitektur RNN/LSTM yang akan
  digunakan. Parameter seperti jumlah \emph{hidden layer}, jumlah
  \emph{units}, penggunaan \emph{dropout layer}, jenis aktivasi, dll.

  Untuk menyederhanakan permasalahan, penggunaan parameter selain yang
  disebutkan dibawah ini menggunakan nilai \texttt{default} dari
  program:

  \begin{itemize}
  \tightlist
  \item
    optimizer: \texttt{adam}
  \item
    activation: \texttt{sigmoid}
  \item
    probability dropout: \texttt{0.1}
  \item
    units: \texttt{20}/lstm-layer
  \item
    loss function: \texttt{mean\ squared\ error}
  \item
    epoch: \texttt{50}
  \item
    batch\_size: \texttt{30}
  \end{itemize}

  Di tahap ini juga dibuat fungsi khusus untuk memperoleh metrik setiap
  epoch. Fungsi khusus yang dibuat antara lain \texttt{nse},
  \texttt{nse\_mod}, dan \texttt{r\_squared}.
\item
  Tahap 5: Evaluasi Model

  Evaluasi yang dilakukan antara lain: melihat perkembangan metrik pada
  setiap epoch, mengevaluasi \emph{train set} dan \emph{test set}.

  Penilaian performa model bergantung pada metrik yang dihasilkan oleh
  \emph{test set}. Metrik yang digunakan sebagai penilaian yaitu
  \emph{mean squared error} (\emph{loss function}), \emph{mean absolute
  error}, \emph{Nash-Sutcliffe Efficiency}, \emph{Modified NSE}, dan
  \emph{Coefficient of Determination}.
\end{enumerate}

    \hypertarget{daftar-pustaka}{%
\section{Daftar Pustaka}\label{daftar-pustaka}}

Berikut daftar pustaka yang berkaitan dengan laporan ``Prediksi Debit
Aliran menggunakan \emph{Long Short-Term Memory} (LSTM)''. Untuk daftar
pustaka yang saya gunakan bisa dilihat pada halaman
github.com/taruma/vivaldi.

    \hypertarget{dataset}{%
\subsection{Dataset}\label{dataset}}

\begin{itemize}
\tightlist
\item
  Megariansyah, Taruma S. (2015): Kajian Penerapan Model NRECA di
  Bendung Pamarayan, Skripsi Program Sarjana, Universitas Katolik
  Parahyangan.
\item
  BMKG (2019): Data Online BMKG, diperoleh melalui situs internet:
  dataonline.bmkg.go.id (diakses pada: 2 Oktober 2019).
\end{itemize}

\hypertarget{makalah-laporan}{%
\subsection{Makalah / Laporan}\label{makalah-laporan}}

\begin{itemize}
\tightlist
\item
  Kratzert, F., Klotz, D., Brenner, C., Schulz, K., Herrnegger, M.,
  2018. Rainfall--runoff modelling using Long Short-Term Memory (LSTM)
  networks. Hydrology and Earth System Sciences 22, 6005--6022.
  https://doi.org/10.5194/hess-22-6005-2018
\item
  LeCun, Y. A., Bottou, L., Orr, G. B., and Müller, K. R.: Efficient
  backprop, Springer, Berlin, Heidelberg, Germany, 2012.
\item
  Minns, A. W. and Hall, M. J.: Artificial neural networks as rainfall-
  runoff models, Hydrolog. Sci. J., 41, 399--417, 1996.
\end{itemize}

\hypertarget{program}{%
\subsection{Program}\label{program}}

\hypertarget{bahasa-pemrograman}{%
\subsubsection{Bahasa Pemrograman}\label{bahasa-pemrograman}}

\begin{itemize}
\tightlist
\item
  Van Rossum, G. \& Drake Jr, F.L., 1995. \emph{Python tutorial},
  Centrum voor Wiskunde en Informatica Amsterdam, The Netherlands.
\end{itemize}

\hypertarget{paket-scipy-scientific-computing-in-python}{%
\subsubsection{Paket Scipy (Scientific Computing in
Python)}\label{paket-scipy-scientific-computing-in-python}}

\begin{itemize}
\tightlist
\item
  Fernando Pérez and Brian E. Granger. IPython: A System for Interactive
  Scientific Computing, Computing in Science \& Engineering, 9, 21-29
  (2007),
  \href{https://doi.org/10.1109/MCSE.2007.53}{DOI:10.1109/MCSE.2007.53}
\item
  John D. Hunter. Matplotlib: A 2D Graphics Environment, Computing in
  Science \& Engineering, 9, 90-95 (2007),
  \href{https://doi.org/10.1109/MCSE.2007.55}{DOI:10.1109/MCSE.2007.55}
\item
  Stéfan van der Walt, S. Chris Colbert and Gaël Varoquaux. The NumPy
  Array: A Structure for Efficient Numerical Computation, Computing in
  Science \& Engineering, 13, 22-30 (2011),
  \href{http://dx.doi.org/10.1109/MCSE.2011.37}{DOI:10.1109/MCSE.2011.37}
\item
  Wes McKinney. Data Structures for Statistical Computing in Python,
  Proceedings of the 9th Python in Science Conference, 51-56 (2010)
\end{itemize}

\hypertarget{jupyter-notebook}{%
\subsubsection{Jupyter Notebook}\label{jupyter-notebook}}

\begin{itemize}
\tightlist
\item
  Thomas, K., Benjamin, R.-K., Fernando, P., Brian, G., Matthias, B.,
  Jonathan, F., \ldots{} Team, J. D. (2016). Jupyter Notebooks -- a
  publishing format for reproducible computational workflows. Stand
  Alone, 87--90. https://doi.org/10.3233/978-1-61499-649-1-87
\end{itemize}

\hypertarget{paket-deep-learning}{%
\subsubsection{Paket Deep Learning}\label{paket-deep-learning}}

\begin{itemize}
\tightlist
\item
  Abadi, Mart'in et al., 2016. Tensorflow: A system for large-scale
  machine learning. In 12th \(USENIX\) Symposium on Operating Systems
  Design and Implementation (\(OSDI\) 16). pp.~265--283.
\item
  Chollet, F.: Keras, available at: https://github.com/keras-team/keras,
  2015.
\end{itemize}

\hypertarget{paket-python}{%
\subsubsection{Paket Python}\label{paket-python}}

\begin{itemize}
\tightlist
\item
  Megariansyah, Taruma. (2019, October 15). hidrokit: Analisis Hidrologi
  dengan Python (Version 0.3.2). Zenodo.
  http://doi.org/10.5281/zenodo.3490672
\item
  Roberts, W., Williams, G., Jackson, E., Nelson, E., Ames, D., 2018.
  Hydrostats: A Python Package for Characterizing Errors between
  Observed and Predicted Time Series. Hydrology 5(4) 66,
  doi:10.3390/hydrology5040066
\end{itemize}

    \hypertarget{daftar-pranala}{%
\section{Daftar Pranala}\label{daftar-pranala}}

Berikut daftar pranala yang disinggung pada laporan.

    \hypertarget{pranala-buku}{%
\subsection{Pranala Buku}\label{pranala-buku}}

\begin{itemize}
\tightlist
\item
  Google Colab:
  https://colab.research.google.com/drive/1bx3ak\_20dcJ7VdGR-djysLIxLaX7pRI2
\item
  Github:
  https://github.com/taruma/vivaldi/blob/master/notebook/github\_taruma\_demo\_lstm\_rr.ipynb
\item
  NBViewer:
  https://nbviewer.jupyter.org/github/taruma/vivaldi/blob/master/notebook/github\_taruma\_demo\_lstm\_rr.ipynb
\item
  Laporan:
  https://github.com/taruma/vivaldi/blob/master/pdf/taruma\_lstm\_rr\_laporan.pdf
\item
  Laporan (Rapih):
  https://github.com/taruma/vivaldi/blob/master/pdf/taruma\_lstm\_rr\_laporan\_rapih.pdf
\item
  Catatan:
  https://github.com/taruma/vivaldi/blob/master/pdf/taruma\_lstm\_rr\_catatan.pdf
\end{itemize}

\hypertarget{catatan}{%
\subsection{Catatan}\label{catatan}}

\begin{itemize}
\tightlist
\item
  Google Colab: https://colab.research.google.com/
\item
  taruma/vivaldi: https://github.com/taruma/vivaldi
\end{itemize}

\hypertarget{panduan}{%
\subsection{Panduan}\label{panduan}}

\begin{itemize}
\tightlist
\item
  hk43:
  https://nbviewer.jupyter.org/gist/taruma/a9dd4ea61db2526853b99600909e9c50
\item
  hk73:
  https://nbviewer.jupyter.org/gist/taruma/b00880905f297013f046dad95dc2e284
\item
  hk53:
  https://nbviewer.jupyter.org/gist/taruma/50460ebfaab5a30c41e7f1a1ac0853e2
\end{itemize}

\hypertarget{tulisan}{%
\subsection{Tulisan}\label{tulisan}}

\begin{itemize}
\tightlist
\item
  https://towardsdatascience.com/smarter-ways-to-encode-categorical-data-for-machine-learning-part-1-of-3-6dca2f71b159
\item
  https://towardsdatascience.com/basic-feature-engineering-to-reach-more-efficient-machine-learning-6294022e17a5
\end{itemize}

\hypertarget{referensi}{%
\subsection{Referensi}\label{referensi}}

\begin{itemize}
\tightlist
\item
  https://scikit-learn.org/stable/modules/generated/sklearn.preprocessing.StandardScaler.html
\item
  https://keras.io/layers/recurrent/
\item
  https://en.wikipedia.org/wiki/Nash--Sutcliffe\_model\_efficiency\_coefficient
\end{itemize}

\hypertarget{hydrostats-hydroerr}{%
\subsection{HydroStats / HydroErr}\label{hydrostats-hydroerr}}

\begin{itemize}
\tightlist
\item
  https://github.com/BYU-Hydroinformatics/HydroErr
\item
  https://github.com/BYU-Hydroinformatics/Hydrostats
\end{itemize}

\hypertarget{license}{%
\subsection{LICENSE}\label{license}}

\begin{itemize}
\tightlist
\item
  MIT: https://github.com/taruma/vivaldi/blob/master/LICENSE
\item
  CC-BY-4.0: https://creativecommons.org/licenses/by/4.0/
\end{itemize}

    \hypertarget{referensi-belajar}{%
\section{Referensi Belajar}\label{referensi-belajar}}

Saya tidak memiliki latar belakang dalam bidang komputer sehingga apa
yang saya pelajari murni dari belajar otodidak yang materi
pembelajarannya diperoleh daring \emph{online}. Jika tertarik daftar
materi pembelajaran saya, bisa lihat di profil koding saya di
\href{https://taruma.github.io/koding}{taruma.github.io/koding} (akan
saya perbarui dengan daftar yang lengkap jika sempat. hehe). Saya hanya
akan menyebutkan beberapa kelas/kursus yang bermanfaat dalam pembuatan
buku ini.

    \hypertarget{belajar-python}{%
\subsection{Belajar Python}\label{belajar-python}}

Saya mempelajari python dimulai dari akhir tahun 2017 sampai sekarang,
jadi masih tergolong awam juga. Saya memulai belajar python dengan
ketertarikan dalam dunia \emph{data science}. Saya mengambil kelas yang
tersedia gratis (\emph{audit access}) di
\href{https://www.edx.org/}{edX.org}. Berikut beberapa kelas yang saya
ambil:

\begin{itemize}
\tightlist
\item
  {[}edX{]} Introduction to Python (DEV236x, DEV274x, DEV330x) oleh
  Microsoft.
\item
  {[}edX{]} Data Science Research Method: Python Edition (DAT273x) oleh
  Microsoft.
\item
  {[}udemy{]} Python for Data Science and Machine Learning Bootcamp oleh
  Jose Portilla.
\item
  {[}edX{]} Using Python for Research oleh HarvardX.
\end{itemize}

    \hypertarget{belajar-machine-learning-deep-learning}{%
\subsection{Belajar Machine Learning / Deep
Learning}\label{belajar-machine-learning-deep-learning}}

Berikut beberapa kelas yang saya ambil terkait deep learning:

\begin{itemize}
\tightlist
\item
  {[}edX{]} Data Science Essentials (DEV203.1x) dan Principle of Machine
  Learning (DEV203.2x) oleh Microsoft.
\item
  {[}udemy{]} Machine Learning A-Z™: Hands-On Python \& R In Data
  Science oleh Kirill Eremenko, Hadelin de Ponteves, SuperDataScience.
\item
  {[}udemy{]} Deep Learning A-Z™: Hands-On Artificial Neural Networks
  oleh Kirill Eremenko, Hadelin de Ponteves, SuperDataScience.
\end{itemize}

    \hypertarget{video-youtube}{%
\subsection{Video Youtube}\label{video-youtube}}

Selain dari kelas juga saya menonton materi dari video youtube. Berikut
daftar video youtube yang membantu saya mempelajari python/machine
learning (Judul video/playlist oleh @nama channel):

\begin{itemize}
\tightlist
\item
  Python Tutorial oleh @Corey Schafer.
\item
  Python Tutorial (Machine Learning with Python, Deep Learning basics
  with Python, Tensorflow and Keras) oleh @sentdex.
\item
  Data Science with Python Pandas by Athena Kan oleh @CS50.
\item
  Roadmap: How to Learn Machine Learning in 6 Months by Zach Miller oleh
  @IDEAS.
\end{itemize}

    \hypertarget{changelog}{%
\section{Changelog}\label{changelog}}

\begin{verbatim}
- 20191022 - 1.0.0 - Initial
\end{verbatim}


    % Add a bibliography block to the postdoc
    
    
    
\end{document}
